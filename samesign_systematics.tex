

\section{Systematic uncertainties determination}  
\label{sec:systematicdetermination}

Several sources of systematic uncertainties have been considered; each of them is described below, together with its estimate. The main sources of systematic uncertainties are estimated
by following the prescriptions of the TopSystematicUncertainties2011.

\subsection{Uncertainties on Monte Carlo samples}

\subsubsection{Trigger and object reconstruction}
The modeling of lepton triggers as well as reconstruction and identification efficiencies in
Monte Carlo simulation are corrected by scale factors, which are a function of lepton kinematics.
These scale factors, as well as their uncertainties, are obtained from studies of $Z \rightarrow \mu \mu$ and $Z \rightarrow e e$ decays.
The uncertainty on the overall yield due uncertainties on trigger, reconstruction, and identification efficiencies 
are obtained by independently shifting these scale factors up and down by one standard deviation.

\subsubsection{Object scales and resolutions}
% jet %~\cite{atlasjets} 
Uncertainties on the efficiency, energy or momentum calibration, and resolution of leptons and jets 
lead to systematic uncertainties on the signal and background acceptances and distributions of discriminating variables.
These uncertainties are summarised in Tables~\ref{tab:systematicsBprime}, \ref{tab:systematicsT53} and
\ref{tab:systematics4t} for signals and in Table~\ref{tab:systematicsBG} for the background samples.
They are evaluated by shifting the energy or momenta of objects by one standard deviation in each direction, 
where the size of each standard deviation associated with each uncertainty is determined by detailed studies.
Shifts for each systematic are applied separately.
%These shifts are propagated to templates of discriminating variables, 
%meaning that the effect of calibration and resolution systematic uncertainties are considered for each bin.

\subsubsection{Luminosity and production cross-sections}
Uncertainty on the measured value of integrated luminosity leads to an uncertainty on the overall scale of backgrounds that are derived using Monte-Carlo.
%These uncertainties affect the overall scale of the various samples.
The uncertainty on the measured luminosity from van der Meer scans was estimated to be 3.7\%. %~\cite{lumi}. 
% Not sure what the most up-to-date reference is for this...
% ATLAS Collaboration, ATLAS-CONF-2011-116, http://cdsweb.cern.ch/record/1376384.
% ATLAS Collaboration, Eur. Phys. J. C71, 1630 (2011).
In addition, uncertainties on the cross-sections of Monte-Carlo samples are considered, 
and these are determined using dedicated studies of Monte-Carlo generators.
Uncertainties on Monte Carlo cross sections are included separately for each background processes considered.
The uncertainty on $t \overline{t}$ processes with an additional boson are taken to be 30\% for a W boson and 50\% for a Z boson\footnote{\url{https://twiki.cern.ch/twiki/bin/viewauth/AtlasProtected/TTplusV}}.
% Cite these references
% Page: https://twiki.cern.ch/twiki/bin/viewauth/AtlasProtected/TTplusV
% Reference: http://arxiv.org/pdf/1204.5678v1.pdf
The uncertainty on the cross-section of $WZ$ and $ZZ$ events is taken to be 5\% with an additional contribution of 24\% per jet added in quadrature included, making the total cross-section uncertainty 34\%.
% From Here: https://twiki.cern.ch/twiki/bin/viewauth/AtlasProtected/TopSystematicUncertainties2011#Systematics_Affecting_Other_MC
% Reference: arXiv:1012.1792
For the $W^{\pm}W^{\pm}$+2 jets sample, the uncertainty on the
cross-section is 50\%.  The uncertainty on the
$t\bar{t}W^{\pm}W^{\mp}$ has been evaluated by setting the renormalization
and factorization scales to the values of 0.5 and 2, and rerunning the
Madgraph event generator in these two configurations. The difference
with respect to the standard configuration has been used to set the up
(+35\%) and down (-24\%) cross-section uncertainties.


\subsubsection{Missing transverse momentum}
When measuring the effect of systematic uncertaities, changes in the momenta of reconstructed objects are applied to the calculation of \met{}.
In addition, uncertainties specific to the calculation of \met{} are considered.  
The effect of Pile-Up on the \met{} resolution is taken to be 6.6\%, 
and uncertainties on \met{} also include contributions arising from calorimeter cells not associated with jets and from soft jets with 7 \GeV\ < \pt\ < 20 \GeV{}.
The latter two uncertainties are not considered to be independent, but rather are applied coherently.

\subsubsection{PDF uncertainty}
Partonic distribution functions (PDFs) along with parton-parton interaction cross sections 
determine hadronic cross sections of inelastic processes in proton-proton collisions according 
to the formula,
%
\begin{equation}
\sigma^{pp\rightarrow X,Y,...}=\sum_{i,j=q(g),\bar{q}(g)} \int f^A_i(f_1,x_1,Q^2) f^B_j(f_2,x_2,Q^2)\sigma^{i j \rightarrow X,Y,...} dx_1dx_2.
\label{eq:QQbarCX}
\end{equation}
%
%\ref{eq:QQbarCX}
Here $\sigma^{i j \rightarrow X,Y,...}$ is a parton-parton cross section of the interacting partons 
$i$-th and $j$-th in incoming protons A and B, $f^A_i(f_1,x_1,Q^2)$ and $ f^B_j(f_2,x_2,Q^2)$ 
represent PDFs of the $i$-th and $j$-th partons, with $f_1$ and $f_2$ being the flavors, $x_1$ and 
$x_2$ momentum fractions of the interacting partons and $Q^2$ an energy scale (transfer). Since PDFs 
are not calculable from first principles but are determined experimentally with uncertainties coming 
from experimental measurements and theoretical models used to extract PDFs, usage of any PDF in the 
calculations introduces an additional systematic uncertainty into the calculated cross section. 
Similarly enters this effect into the estimate of the signal event selection efficiency from a Monte-Carlo 
(MC) simulation.


\vspace{0.4cm}
\noindent
{\bf Method:} 
PDF systematics uncertainty estimation in this analysis is done using {\bf Reweighting method}~\cite{PDF_RW}.
The idea of the method is as follows: suppose, we have a sample of events modeling hard scattering 
interactions in proton-proton collisions produced using an MC event generator. Let $PDF^0$ be a PDF 
used in the generator to produce these events. If we apply some selection cuts to these events, then 
the signal event selection efficiency, $\epsilon^0$, by definition is,
%
\begin{equation}
\epsilon^0 = \frac{N_{cuts}^0}{N_{gen}^0},
\label{eq:eff}
\end{equation}
%
where $N_{gen}^0$ is the total number of events generated by the generator before applying any cut and 
$N_{cuts}^0$ is the number of selected signal events after applying all selection criteria (including 
object and event selections). If an event filter of an efficiency $\epsilon^0_{gen}$ is applied to 
the generator, then the event selection efficiency will look like 
%
\begin{equation}
\label{eq:epsilon}
\epsilon^0 =\epsilon_{gen}^0 \cdot \frac{N_{cuts}^0}{N_{ef}^0}.
\end{equation}
%

Now, had these events been generated with the same generator but using another PDF (let call it $PDF^l$), 
then the relative probability of producing a particular event $e_i$ as a result of an interaction of two 
partons with the same flavor, momentum fraction and energy scale, as in the case of using the default PDF 
$PDF^0$, will be defined, as obvious from the formula~\eqref{eq:QQbarCX}, by the relative PDFs weight, 
$w^i$, defined as
% 
\begin{equation}
\label{eq:PDFrw}
w^i=\frac{f_{PDF^l}(f_1,x_1,Q^2)}{f_{PDF^0}(f_1,x_1,Q^2)} \times \frac{f_{PDF^l}(f_2,x_2,Q^2)}{f_{PDF^0}(f_2,x_2,Q^2)}.
\end{equation}
%

$N^l_{cuts}$ and $N^l_{gen}$ relevant to the sample produced using PDF $PDF^l$ in the generator will be 
defined as $N^l_{cuts} = \sum^{N^0_{cuts}}_{i=1}w^i $ and $N^l_{gen} = \sum^{N^0_{gen}}_{i=1}w^i $ and the 
event selection efficiency will look like
%
\[ 
\epsilon^l = \epsilon^l_{gen} \cdot \frac{N^l_{cuts}}{N^l_{gen}} = \epsilon^l_{gen} \cdot \frac{ \sum^{N^1_{cuts}}_{i=1}w^i }{ \sum^{N^1_{gen}}_{i=1}w^i}. 
\]
%
The generator filter efficiencies $\epsilon^0_{gen}$ and $\epsilon^l_{gen}$ have to be determined by separate 
runs of the generator including the PDFs $PDF^0$ and $PDF^l$ into the generator each time. 

The systematic uncertainty on the event selection efficiency is then $\delta \epsilon^l = \epsilon^l-\epsilon^0$ 
and the corresponding relative uncertainty is
%
\begin{equation}
\label{eq:}
\frac{\delta \epsilon^l}{\epsilon^0} =\frac{\epsilon^0 - \epsilon^l}{\epsilon^0}.
\end{equation}
%

To extract the necessary PDF info of the signal events from samples (ntuple level) and to calculate the corresponding 
relative per event PDF weights, $w^i$ (formula~\ref{eq:PDFrw}), the {\bf TopPdfUncertainty 
package}~\cite{TopPdfUncert} from the ATLAS Top working group has been used.


\vspace{0.4cm}
\noindent
{\bf PDF choice:}
The PDF systematics study has been applied to an analysis searching for fourth generation down-type 
quark $b'$ (a mass point of 500~GeV) in the process 
$pp \rightarrow b'\bar{b'}\rightarrow tW^-\bar{t}W^+ \rightarrow bW^+W^-\bar{b}W^-W^+$ with two same 
sign ($+ +$ or $- -$) leptons (electrons or muons in the combinations $e e$, $e \mu$ and $\mu \mu$) 
in the final state. The signal samples used in this analysis have been generated with the Pythia6 
generator using the ATLAS MC11c production setup. In this setup MRST LO** PDF (LHAPDF set number 
20651)~\cite{LOstar,LOstarstar} is the default PDF and the "ATLAS Underlying Event Tune 2B" (AUET2B)
%(ATL-PHYS-PUB-2011-009) 
tune~\cite{AUET2B} is the corresponding UE tune. MRST LO** is an LO generator, thus the following 
alternative LO PDFs have been used in this study:
\begin{itemize}
%
\item CTEQ6L1 (10042) - with LO fit and LO $\alpha_S$~\cite{CTEQ6};
%
\item MSTW2008LO (21000)~\cite{MSTW2008LO};
%
\item MRST LO* (20650)~\cite{MRSTLOX};
%
\item CT09MC2 (10772)~\cite{CT09MC2};
%
\end{itemize}


\vspace{0.4cm}
\noindent
{\bf Results:}
The uppermost and lowest shifts in the signal event selection efficiency extracted from the reweighting method
considering all, $e$-$e$, $e$-$\mu$ and $\mu$-$\mu$, channels are $+1.4$\% and $-1.1$\% , with the average 
shifts being around 0.6\%.


\subsubsection{Initial and final state radiation}
Initial and final state radiation (ISR, FSR) will generate additional
jets, through the emission of gluons from the initial and final legs
in the Feynman diagrams. To evaluate the uncertainty of this effect,
dedicated samples were produced, by changing the parton shower (PS)
parameters in Pythia, as reported in table~\ref{tab:ISFFRS}.
\begin{table}[htbp]
\begin{center}
\caption{\textit{Pythia parameters changed to produce samples with more and less parton shower. Those samples were used to estimate ISR, FSR systematic uncertainty.}}
\label{tab:ISFFRS}
\begin{tabular}{c|c|c|l}
\hline\hline
Pythia parameter & Less PS & More PS & Comment\\
PARP(67) & 0.70   & 1.75   & controls high-pt ISR branchings phase-space \\
PARP(64) & 3.60   & 0.60   & multiplicative factor of the momentum scale squared\\
         &        &        & in running $\alpha_s$, used in ISR \\
PARP(72) & 0.2150 & 0.6450 & multiplicative factor of the $\Lambda_{QCD}$\\
         &        &        & in running $\alpha_s$, used in FSR \\
PARJ(82) & 1.66   & 0.5    & FSR low-pt cutoff \\
\hline
\end{tabular}
\end{center}
\end{table}
The full analysis was run on the 150k events done for each dataset, with more or less PS, at the b' mass point of 500~GeV.
By observing the changes in the yields, the ISR/FSR systematic uncertainty has been evaluated of the order of 6\%.

\subsubsection{Systematic uncertainties on $b^\prime$ signals}
\begin{centering}
\tiny
Channel: $ee$ \\
\begin{tabular}{r|p{.06\linewidth}p{.06\linewidth}p{.06\linewidth}p{.06\linewidth}p{.06\linewidth}p{.06\linewidth}p{.06\linewidth}p{.06\linewidth}p{.06\linewidth}}
\toprule
 Sys.  & $b'$ 400GeV & $b'$ 450GeV & $b'$ 500GeV & $b'$ 550GeV & $b'$ 600GeV & $b'$ 650GeV & $b'$ 700GeV & $b'$ 750GeV & $b'$ 800GeV \\
\toprule
JES  & 9.5/-9.0 & 3.9/-5.7 & 3.0/-4.8 & 3.4/-2.9 & 2.1/-1.6 & 0.7/-1.7 & 0.9/-0.3 & 0.5/-0.6 & 1.1/-2.2 \\
JER  & 0.8/-0.9 & 0.9/-1.0 & 0.1/-0.2 & 0.7/-0.8 & 0.2/-0.3 & 0.5/-0.6 & 0.2/-0.3 & 0.4/-0.5 & 0.3/-0.4 \\
BJET  & 1.9/-1.7 & 1.9/-1.9 & 0.0/-1.6 & 1.0/-0.8 & 0.5/-1.0 & 0.0/-0.6 & 0.5/0.0 & 0.3/-0.4 & 0.4/-0.5 \\
JVF  & 3.1/-3.1 & 3.0/-2.9 & 2.9/-2.9 & 2.9/-2.8 & 2.9/-2.9 & 2.8/-2.8 & 2.9/-2.8 & 2.9/-2.8 & 2.9/-2.8 \\
SoftJets  & 0.0/0.0 & 0.3/0.0 & 0.0/-0.5 & 0.1/0.0 & 0.1/-0.1 & 0.0/0.0 & 0.4/0.0 & 0.0/0.0 & 0.0/-0.1 \\
MET  & 0.0/0.0 & 0.0/0.0 & 0.0/-0.4 & 0.1/0.0 & 0.1/0.0 & 0.0/0.0 & 0.4/0.0 & 0.0/0.0 & 0.0/0.0 \\
JEFF  & 0.4/-0.5 & 0.1/-0.2 & 0.1/-0.2 & 0.0/-0.1 & 0.0/-0.1 & 0.1/-0.2 & 0.0/-0.1 & 0.0/-0.1 & 0.0/-0.1 \\
EES  & 1.6/0.0 & 2.0/0.0 & 0.4/-0.5 & 0.7/0.0 & 1.3/0.0 & 1.3/-0.4 & 0.4/-0.2 & 0.8/0.0 & 0.3/-0.2 \\
EER  & 0.0/0.0 & 0.2/0.0 & 0.0/-1.4 & 0.0/0.0 & 0.2/0.0 & 0.0/-0.3 & 0.4/-0.3 & 0.4/0.0 & 0.0/0.0 \\
MER  & 0.4/0.0 & 0.0/0.0 & 0.0/-0.4 & 0.0/-0.3 & 0.0/0.0 & 0.0/-0.2 & 0.0/0.0 & 0.0/0.0 & 0.0/0.0 \\
MES  & 0.0/0.0 & 0.0/-0.1 & 0.0/-0.1 & 0.0/-0.1 & 0.0/-0.1 & 0.0/-0.1 & 0.0/-0.1 & 0.0/-0.1 & 0.0/-0.1 \\
MuTSF  & 0.1/-0.2 & 0.1/-0.2 & 0.1/-0.2 & 0.2/-0.3 & 0.1/-0.2 & 0.2/-0.3 & 0.1/-0.2 & 0.1/-0.2 & 0.2/-0.3 \\
ElTSF  & 1.0/-1.1 & 1.0/-1.1 & 1.0/-1.1 & 1.0/-1.1 & 1.0/-1.1 & 1.0/-1.1 & 1.0/-1.1 & 1.0/-1.1 & 1.0/-1.1 \\
ElRecoSF  & 1.9/-2.0 & 1.9/-1.9 & 1.9/-1.9 & 1.8/-1.9 & 1.9/-1.9 & 1.9/-2.0 & 1.9/-1.9 & 1.8/-1.9 & 1.9/-1.9 \\
ElIDSF  & 4.7/-4.7 & 4.6/-4.6 & 4.7/-4.7 & 4.7/-4.6 & 4.7/-4.7 & 4.7/-4.7 & 4.7/-4.7 & 4.7/-4.7 & 4.7/-4.6 \\
MuRecoSF  & 0.0/-0.1 & 0.0/-0.1 & 0.0/-0.1 & 0.0/-0.1 & 0.0/-0.1 & 0.0/-0.1 & 0.0/-0.1 & 0.0/-0.1 & 0.0/-0.1 \\
MuIDSF  & 0.0/-0.1 & 0.0/-0.1 & 0.1/-0.2 & 0.1/-0.2 & 0.0/-0.1 & 0.1/-0.2 & 0.0/-0.1 & 0.0/-0.1 & 0.1/-0.2 \\
 \\
\bottomrule
\end{tabular}

\\
Channel: $e \mu$ \\
\begin{tabular}{r|p{.08\linewidth}p{.08\linewidth}p{.08\linewidth}p{.08\linewidth}p{.08\linewidth}p{.08\linewidth}p{.08\linewidth}p{.08\linewidth}p{.08\linewidth}}
\toprule
 Sys.  & $b'$ 400GeV & $b'$ 450GeV & $b'$ 500GeV & $b'$ 550GeV & $b'$ 600GeV & $b'$ 650GeV & $b'$ 700GeV & $b'$ 750GeV & $b'$ 800GeV \\
\toprule
JES  & 6.3/-9.0 & 4.0/-4.2 & 3.4/-3.4 & 2.4/-3.0 & 1.1/-1.9 & 0.6/-1.1 & 0.7/-0.4 & 0.7/-0.2 & 0.3/0.0 \\
JER  & 0.2/-0.3 & 0.1/-0.2 & 0.0/-0.1 & 1.0/-1.1 & 0.0/-0.1 & 0.0/-0.1 & 0.1/-0.2 & 0.0/-0.1 & 0.2/-0.3 \\
BJET  & 2.1/-1.6 & 1.2/-1.3 & 0.7/-0.7 & 0.5/-0.7 & 0.3/-0.3 & 0.3/-0.7 & 0.5/-0.4 & 0.5/-0.3 & 0.2/-0.3 \\
JVF  & 3.1/-3.1 & 3.0/-2.9 & 3.0/-3.0 & 3.0/-2.9 & 2.9/-2.8 & 2.9/-2.9 & 2.9/-2.9 & 2.8/-2.8 & 2.8/-2.8 \\
CellOutSoftJet  & 0.0/0.0 & 0.0/-0.3 & 0.0/-0.1 & 0.0/-0.1 & 0.0/0.0 & 0.0/-0.1 & 0.0/0.0 & 0.0/-0.1 & 0.0/-0.1 \\
METPileUp  & 0.0/0.0 & 0.0/-0.1 & 0.1/-0.1 & 0.0/-0.1 & 0.1/0.0 & 0.0/-0.2 & 0.0/0.0 & 0.0/-0.1 & 0.0/0.0 \\
JEFF  & 0.1/-0.2 & 0.0/-0.1 & 0.0/-0.1 & 0.0/-0.1 & 0.0/-0.1 & 0.0/-0.1 & 0.0/-0.1 & 0.0/0.0 & 0.1/-0.2 \\
EES  & 0.9/0.0 & 1.3/0.0 & 0.6/0.0 & 0.5/0.0 & 0.2/-0.1 & 0.0/0.0 & 0.5/0.0 & 0.2/-0.2 & 0.2/-0.4 \\
EER  & 0.0/0.0 & 0.0/0.0 & 0.1/0.0 & 0.2/-0.1 & 0.0/0.0 & 0.0/-0.2 & 0.1/0.0 & 0.0/0.0 & 0.0/-0.4 \\
MER  & 0.3/-0.2 & 0.2/0.0 & 0.0/0.0 & 0.3/-0.5 & 0.0/-0.2 & 0.0/-0.3 & 0.0/-0.2 & 0.0/-0.2 & 0.1/-0.1 \\
MES  & 0.0/0.0 & 0.0/-0.1 & 0.0/-0.1 & 0.0/-0.1 & 0.0/-0.1 & 0.0/-0.1 & 0.0/0.0 & 0.0/-0.1 & 0.0/-0.1 \\
MuTSF  & 1.5/-1.6 & 1.4/-1.5 & 1.5/-1.6 & 1.5/-1.6 & 1.5/-1.6 & 1.5/-1.6 & 1.6/-1.7 & 1.5/-1.6 & 1.5/-1.6 \\
ElTSF  & 0.5/-0.6 & 0.5/-0.6 & 0.5/-0.6 & 0.5/-0.6 & 0.5/-0.6 & 0.5/-0.6 & 0.5/-0.6 & 0.5/-0.6 & 0.5/-0.6 \\
ElRecoSF  & 0.9/-1.0 & 0.9/-1.0 & 0.9/-1.0 & 0.9/-1.0 & 0.9/-1.0 & 1.0/-1.1 & 1.0/-1.1 & 1.0/-1.1 & 1.0/-1.1 \\
ElIDSF  & 2.3/-2.4 & 2.4/-2.5 & 2.4/-2.5 & 2.4/-2.5 & 2.4/-2.5 & 2.5/-2.6 & 2.4/-2.5 & 2.4/-2.5 & 2.4/-2.5 \\
MuRecoSF  & 0.3/-0.4 & 0.3/-0.4 & 0.3/-0.4 & 0.4/-0.5 & 0.4/-0.5 & 0.4/-0.5 & 0.4/-0.5 & 0.4/-0.5 & 0.4/-0.5 \\
MuIDSF  & 0.8/-0.9 & 0.8/-0.9 & 0.8/-0.9 & 0.8/-0.9 & 0.8/-0.9 & 0.8/-0.9 & 0.8/-0.9 & 0.8/-0.9 & 0.8/-0.9 \\
 \\
\bottomrule
\end{tabular}

\\
Channel: $\mu \mu$ \\
\begin{tabular}{r|p{.08\linewidth}p{.08\linewidth}p{.08\linewidth}p{.08\linewidth}p{.08\linewidth}p{.08\linewidth}p{.08\linewidth}p{.08\linewidth}p{.08\linewidth}}
\toprule
 Sys.  & $b'$ 400GeV & $b'$ 450GeV & $b'$ 500GeV & $b'$ 550GeV & $b'$ 600GeV & $b'$ 650GeV & $b'$ 700GeV & $b'$ 750GeV & $b'$ 800GeV \\
\toprule
JES  & 4.5/-6.6 & 3.2/-5.4 & 2.2/-4.6 & 1.3/-1.1 & 1.0/-1.6 & 0.0/-0.2 & 0.1/-1.1 & 0.0/0.0 & 0.0/0.0 \\
JER  & 0.7/-0.8 & 0.3/-0.4 & 0.5/-0.6 & 0.0/-0.1 & 0.1/-0.2 & 0.5/-0.6 & 0.7/-0.8 & 0.0/-0.1 & 0.3/-0.4 \\
BJET  & 1.6/-0.5 & 0.8/-1.4 & 0.0/-0.6 & 0.8/-0.4 & 1.0/-0.8 & 0.2/-0.3 & 0.1/-0.2 & 0.1/0.0 & 0.1/-0.4 \\
JVF  & 3.2/-3.1 & 3.1/-3.0 & 2.9/-2.8 & 3.0/-2.9 & 3.0/-2.9 & 2.9/-2.9 & 2.9/-2.9 & 2.9/-2.8 & 2.9/-2.8 \\
CellOutSoftJet  & 0.6/0.0 & 0.7/-0.2 & 0.2/-0.3 & 0.0/0.0 & 0.0/0.0 & 0.0/0.0 & 0.0/0.0 & 0.0/0.0 & 0.0/0.0 \\
METPileUp  & 0.2/-0.1 & 0.7/0.0 & 0.0/-0.1 & 0.0/-0.1 & 0.1/0.0 & 0.0/-0.1 & 0.0/0.0 & 0.0/0.0 & 0.0/0.0 \\
JEFF  & 0.1/-0.2 & 0.1/-0.2 & 0.0/-0.1 & 0.1/-0.2 & 0.0/-0.1 & 0.1/-0.2 & 0.0/-0.1 & 0.0/-0.1 & 0.0/-0.1 \\
EES  & 0.0/-0.2 & 0.3/0.0 & 0.0/-0.1 & 0.0/0.0 & 0.1/0.0 & 0.0/-0.2 & 0.0/-0.1 & 0.0/0.0 & 0.0/0.0 \\
EER  & 0.0/-0.3 & 0.0/0.0 & 0.0/-0.1 & 0.0/0.0 & 0.1/0.0 & 0.0/-0.2 & 0.0/-0.1 & 0.0/0.0 & 0.0/0.0 \\
MER  & 0.2/0.0 & 0.0/-1.0 & 0.1/-0.3 & 0.0/-0.4 & 0.0/-0.5 & 0.2/-0.3 & 0.4/-0.1 & 0.1/-0.1 & 0.3/-0.1 \\
MES  & 0.0/-0.1 & 0.0/0.0 & 0.0/-0.1 & 0.0/0.0 & 0.0/-0.1 & 0.1/-0.2 & 0.0/0.0 & 0.0/-0.1 & 0.0/-0.1 \\
MuTSF  & 2.8/-2.8 & 2.8/-2.9 & 2.8/-2.9 & 2.8/-2.9 & 3.0/-2.9 & 2.7/-3.0 & 2.9/-2.9 & 2.8/-2.8 & 2.8/-2.9 \\
ElTSF  & 0.0/-0.1 & 0.0/-0.1 & 0.0/-0.1 & 0.0/-0.1 & 0.0/-0.1 & 0.0/-0.2 & 0.0/-0.1 & 0.0/-0.1 & 0.0/-0.1 \\
ElRecoSF  & 0.0/-0.1 & 0.1/-0.2 & 0.0/-0.2 & 0.0/-0.1 & 0.0/0.0 & 0.0/-0.1 & 0.0/-0.1 & 0.0/-0.1 & 0.0/-0.1 \\
ElIDSF  & 0.2/-0.3 & 0.2/-0.3 & 0.2/-0.4 & 0.2/-0.3 & 0.3/-0.3 & 0.1/-0.4 & 0.1/-0.2 & 0.1/-0.2 & 0.2/-0.3 \\
MuRecoSF  & 0.7/-0.8 & 0.7/-0.8 & 0.6/-0.9 & 0.7/-0.8 & 0.9/-0.8 & 0.6/-0.9 & 0.7/-0.8 & 0.7/-0.8 & 0.8/-0.8 \\
MuIDSF  & 1.5/-1.6 & 1.5/-1.6 & 1.4/-1.7 & 1.5/-1.6 & 1.7/-1.5 & 1.5/-1.7 & 1.5/-1.6 & 1.5/-1.6 & 1.5/-1.6 \\
 \\
\bottomrule
\end{tabular}

\\
\label{tab:systematicsBprime}
\end{centering}


\subsubsection{Systematic uncertainties on $T_{5/3}$ signals}
\begin{centering}
\tiny
Channel: $ee$ \\
\begin{tabular}{r|p{.08\linewidth}p{.08\linewidth}p{.08\linewidth}p{.08\linewidth}p{.08\linewidth}p{.08\linewidth}p{.08\linewidth}p{.08\linewidth}}
\toprule
 Sys.  & 450GeV $\lambda=1$  & 450GeV $\lambda=3$  & 550GeV $\lambda=1$  & 550GeV $\lambda=3$  & 650GeV $\lambda=1$  & 650GeV $\lambda=3$  & 750GeV $\lambda=1$  & 750GeV $\lambda=3$  \\
\toprule
JES  & 5.1/-7.1 & 5.6/-5.4 & 2.2/-2.8 & 4.1/-3.9 & 0.9/-2.0 & 2.5/-3.7 & 1.1/-2.3 & 1.7/-2.0 \\
JER  & 1.0/-1.1 & 0.8/-0.9 & 0.6/-0.7 & 0.1/-0.2 & 0.4/-0.5 & 0.0/-0.1 & 0.0/-0.1 & 0.0/-0.1 \\
BJET  & 1.2/-2.4 & 0.0/-1.4 & 0.8/-0.3 & 0.7/-0.8 & 0.9/-0.7 & 0.3/-0.9 & 0.5/-1.1 & 0.7/-0.8 \\
JVF  & 3.0/-2.9 & 3.0/-2.9 & 2.9/-2.9 & 2.9/-2.9 & 2.9/-2.9 & 2.8/-2.8 & 2.9/-2.8 & 2.7/-2.7 \\
CellOutSoftJet  & 0.0/-0.1 & 0.3/-0.4 & 0.0/0.0 & 0.2/-0.3 & 0.0/-0.2 & 0.0/0.0 & 0.0/-0.3 & 0.1/0.0 \\
METPileUp  & 0.0/0.0 & 0.1/-0.4 & 0.0/0.0 & 0.1/-0.3 & 0.0/-0.2 & 0.0/0.0 & 0.0/-0.3 & 0.0/0.0 \\
JEFF  & 0.1/-0.2 & 0.0/-0.1 & 0.1/-0.2 & 0.0/-0.1 & 0.0/-0.1 & 0.1/-0.2 & 0.0/-0.1 & 0.1/-0.2 \\
EES  & 1.2/0.0 & 0.8/0.0 & 0.5/0.0 & 0.3/-0.4 & 0.3/-0.4 & 0.0/-0.1 & 0.6/-0.3 & 0.8/0.0 \\
EER  & 0.0/0.0 & 0.0/-0.5 & 0.2/-0.2 & 0.0/0.0 & 0.0/0.0 & 0.0/-0.3 & 0.0/0.0 & 0.0/-0.2 \\
MER  & 0.0/-0.3 & 0.1/-0.2 & 0.1/-0.2 & 0.0/-0.2 & 0.1/-0.2 & 0.0/-0.2 & 0.0/-0.2 & 0.0/0.0 \\
MES  & 0.0/0.0 & 0.0/0.0 & 0.0/-0.1 & 0.0/-0.1 & 0.0/-0.1 & 0.0/-0.1 & 0.0/-0.1 & 0.0/0.0 \\
MuTSF  & 0.1/-0.2 & 0.1/-0.2 & 0.2/-0.3 & 0.1/-0.2 & 0.1/-0.2 & 0.2/-0.3 & 0.2/-0.3 & 0.1/-0.2 \\
ElTSF  & 1.0/-1.1 & 1.0/-1.1 & 1.0/-1.1 & 1.0/-1.1 & 1.0/-1.1 & 1.0/-1.1 & 1.0/-1.1 & 1.0/-1.1 \\
ElRecoSF  & 1.9/-1.9 & 1.8/-1.9 & 1.9/-1.9 & 1.8/-1.9 & 1.9/-1.9 & 1.9/-1.9 & 1.8/-1.9 & 1.9/-1.9 \\
ElIDSF  & 4.7/-4.7 & 4.7/-4.6 & 4.7/-4.7 & 4.6/-4.6 & 4.7/-4.7 & 4.7/-4.7 & 4.7/-4.7 & 4.7/-4.6 \\
MuRecoSF  & 0.0/-0.1 & 0.0/-0.1 & 0.0/-0.1 & 0.0/-0.1 & 0.0/-0.1 & 0.0/-0.1 & 0.0/-0.1 & 0.0/-0.1 \\
MuIDSF  & 0.0/-0.1 & 0.1/-0.2 & 0.1/-0.2 & 0.1/-0.2 & 0.1/-0.2 & 0.0/-0.1 & 0.1/-0.2 & 0.0/-0.1 \\
 \\
\bottomrule
\end{tabular}

\\
Channel: $e \mu$ \\
\begin{tabular}{r|p{.08\linewidth}p{.08\linewidth}p{.08\linewidth}p{.08\linewidth}p{.08\linewidth}p{.08\linewidth}p{.08\linewidth}p{.08\linewidth}}
\toprule
 Sys.  & 450GeV $\lambda=1$  & 450GeV $\lambda=3$  & 550GeV $\lambda=1$  & 550GeV $\lambda=3$  & 650GeV $\lambda=1$  & 650GeV $\lambda=3$  & 750GeV $\lambda=1$  & 750GeV $\lambda=3$  \\
\toprule
JES  & 3.9/-5.7 & 4.6/-4.5 & 3.0/-2.7 & 3.1/-3.5 & 0.8/-1.5 & 2.5/-3.5 & 0.5/-1.3 & 2.8/-1.9 \\
JER  & 0.4/-0.5 & 0.0/-0.1 & 0.1/-0.2 & 0.3/-0.4 & 0.3/-0.4 & 0.3/-0.4 & 0.2/-0.3 & 0.0/-0.1 \\
BJET  & 0.6/-1.1 & 1.1/-0.9 & 0.8/-1.0 & 0.8/-1.0 & 0.6/-0.5 & 0.7/-1.1 & 0.3/-0.4 & 0.8/-0.6 \\
JVF  & 3.0/-3.0 & 3.1/-3.0 & 3.0/-2.9 & 2.9/-2.8 & 3.0/-2.9 & 2.8/-2.7 & 2.9/-2.9 & 2.7/-2.6 \\
SoftJets  & 0.1/-0.1 & 0.1/0.0 & 0.0/-0.1 & 0.0/-0.2 & 0.0/-0.1 & 0.0/0.0 & 0.1/0.0 & 0.1/-0.2 \\
MET  & 0.2/-0.1 & 0.2/0.0 & 0.0/-0.2 & 0.0/-0.1 & 0.0/-0.1 & 0.0/0.0 & 0.0/0.0 & 0.1/-0.1 \\
JEFF  & 0.1/-0.2 & 0.0/-0.1 & 0.0/-0.1 & 0.0/-0.1 & 0.1/-0.2 & 0.1/-0.2 & 0.0/-0.1 & 0.0/-0.1 \\
EES  & 0.9/0.0 & 0.3/-0.2 & 0.2/-0.3 & 0.4/0.0 & 0.4/0.0 & 0.1/-0.1 & 0.2/0.0 & 0.2/-0.1 \\
EER  & 0.0/0.0 & 0.0/-0.1 & 0.0/0.0 & 0.0/0.0 & 0.1/-0.1 & 0.0/-0.2 & 0.0/0.0 & 0.0/0.0 \\
MER  & 0.0/-0.2 & 0.1/-0.1 & 0.2/-0.1 & 0.1/-0.2 & 0.0/-0.2 & 0.0/-0.1 & 0.0/-0.1 & 0.2/-0.2 \\
MES  & 0.0/-0.1 & 0.0/0.0 & 0.0/0.0 & 0.0/-0.1 & 0.0/-0.1 & 0.0/-0.1 & 0.0/-0.1 & 0.0/-0.1 \\
MuTSF  & 1.5/-1.6 & 1.5/-1.6 & 1.5/-1.6 & 1.4/-1.5 & 1.5/-1.6 & 1.5/-1.6 & 1.5/-1.6 & 1.5/-1.6 \\
ElTSF  & 0.5/-0.6 & 0.5/-0.6 & 0.5/-0.6 & 0.5/-0.6 & 0.5/-0.6 & 0.5/-0.6 & 0.5/-0.6 & 0.5/-0.6 \\
ElRecoSF  & 0.9/-1.0 & 0.9/-1.0 & 1.0/-1.1 & 1.0/-1.1 & 1.0/-1.1 & 0.9/-1.0 & 1.0/-1.1 & 0.9/-1.0 \\
ElIDSF  & 2.4/-2.5 & 2.4/-2.5 & 2.4/-2.5 & 2.5/-2.6 & 2.5/-2.5 & 2.4/-2.5 & 2.4/-2.5 & 2.4/-2.5 \\
MuRecoSF  & 0.3/-0.4 & 0.3/-0.4 & 0.4/-0.5 & 0.3/-0.4 & 0.4/-0.5 & 0.4/-0.5 & 0.4/-0.5 & 0.4/-0.5 \\
MuIDSF  & 0.8/-0.9 & 0.8/-0.9 & 0.8/-0.9 & 0.8/-0.9 & 0.8/-0.9 & 0.8/-0.9 & 0.8/-0.9 & 0.8/-0.9 \\
 \\
\bottomrule
\end{tabular}

\\
Channel: $\mu \mu$ \\
\begin{tabular}{r|p{.08\linewidth}p{.08\linewidth}p{.08\linewidth}p{.08\linewidth}p{.08\linewidth}p{.08\linewidth}p{.08\linewidth}p{.08\linewidth}}
\toprule
 Sys.  & 450GeV $\lambda=1$  & 450GeV $\lambda=3$  & 550GeV $\lambda=1$  & 550GeV $\lambda=3$  & 650GeV $\lambda=1$  & 650GeV $\lambda=3$  & 750GeV $\lambda=1$  & 750GeV $\lambda=3$  \\
\toprule
JES  & 4.2/-4.0 & 4.9/-4.5 & 1.4/-1.3 & 2.2/-2.9 & 0.2/-0.9 & 1.1/-2.3 & 0.4/-0.5 & 1.7/-0.8 \\
JER  & 0.0/-0.1 & 0.3/-0.4 & 0.0/-0.1 & 0.6/-0.7 & 0.1/-0.2 & 0.2/-0.3 & 0.3/-0.4 & 0.1/-0.2 \\
BJET  & 0.7/-0.4 & 0.7/-0.8 & 0.6/-0.8 & 0.6/-1.4 & 0.3/-0.7 & 0.6/-0.8 & 0.2/-0.4 & 1.5/-0.8 \\
JVF  & 3.1/-3.0 & 3.0/-3.0 & 3.0/-3.0 & 2.9/-2.9 & 2.9/-2.9 & 2.8/-2.8 & 2.9/-2.8 & 2.7/-2.7 \\
CellOutSoftJet  & 0.0/0.0 & 0.3/0.0 & 0.0/-0.2 & 0.0/0.0 & 0.0/0.0 & 0.2/0.0 & 0.0/-0.1 & 0.0/-0.1 \\
METPileUp  & 0.0/0.0 & 0.3/-0.1 & 0.0/-0.2 & 0.0/0.0 & 0.0/0.0 & 0.1/-0.1 & 0.0/-0.2 & 0.0/-0.1 \\
JEFF  & 0.1/-0.2 & 0.0/-0.1 & 0.0/-0.1 & 0.0/-0.1 & 0.0/-0.1 & 0.2/-0.3 & 0.0/-0.1 & 0.0/-0.1 \\
EES  & 0.1/-0.1 & 0.0/0.0 & 0.0/-0.1 & 0.1/0.0 & 0.0/0.0 & 0.0/0.0 & 0.0/0.0 & 0.0/0.0 \\
EER  & 0.0/-0.1 & 0.0/0.0 & 0.0/-0.1 & 0.0/0.0 & 0.0/-0.1 & 0.0/-0.1 & 0.0/0.0 & 0.0/-0.1 \\
MER  & 0.7/-0.8 & 0.3/0.0 & 0.3/-0.1 & 0.0/-0.5 & 0.1/-0.2 & 0.3/-0.4 & 0.0/-0.3 & 0.1/-0.5 \\
MES  & 0.0/-0.1 & 0.0/-0.1 & 0.0/-0.1 & 0.0/-0.1 & 0.0/-0.1 & 0.0/-0.1 & 0.0/-0.1 & 0.0/0.0 \\
MuTSF  & 2.8/-2.9 & 2.8/-2.8 & 2.8/-2.8 & 2.7/-2.7 & 2.9/-2.9 & 2.8/-2.9 & 2.9/-3.0 & 2.8/-2.9 \\
ElTSF  & 0.0/-0.1 & 0.0/-0.1 & 0.0/-0.1 & 0.0/-0.1 & 0.0/-0.1 & 0.0/-0.1 & 0.0/-0.1 & 0.0/-0.1 \\
ElRecoSF  & 0.0/-0.1 & 0.0/-0.1 & 0.1/-0.1 & 0.0/-0.1 & 0.1/-0.2 & 0.0/-0.1 & 0.1/-0.2 & 0.0/-0.1 \\
ElIDSF  & 0.2/-0.3 & 0.2/-0.3 & 0.2/-0.3 & 0.2/-0.3 & 0.2/-0.3 & 0.2/-0.3 & 0.2/-0.3 & 0.1/-0.2 \\
MuRecoSF  & 0.7/-0.8 & 0.7/-0.8 & 0.7/-0.8 & 0.7/-0.8 & 0.7/-0.8 & 0.8/-0.9 & 0.8/-0.9 & 0.7/-0.8 \\
MuIDSF  & 1.5/-1.6 & 1.5/-1.6 & 1.5/-1.6 & 1.5/-1.6 & 1.5/-1.6 & 1.5/-1.6 & 1.5/-1.6 & 1.5/-1.6 \\
 \\
\bottomrule
\end{tabular}

\\
\label{tab:systematicsT53}
\end{centering}


\subsubsection{Systematic uncertainties on four top quarks signal}
\begin{centering}
\tiny
Channel: $ee$ \\
\begin{tabular}{r|p{.08\linewidth}}
\toprule
 Sys.  & FourTops \\
\toprule
JES  & 4.2/-1.4 \\
JER  & 0.8/-0.9 \\
BJET  & 1.3/0.0 \\
JVF  & 4.0/-3.9 \\
SoftJets  & 0.0/0.0 \\
MET  & 0.0/0.0 \\
JEFF  & 0.0/-0.1 \\
EES  & 2.7/0.0 \\
EER  & 0.0/0.0 \\
MER  & 0.4/0.0 \\
MES  & 0.0/-0.1 \\
MuTSF  & 0.2/-0.3 \\
ElTSF  & 1.0/-1.1 \\
ElRecoSF  & 1.9/-1.9 \\
ElIDSF  & 4.7/-4.7 \\
MuRecoSF  & 0.0/-0.1 \\
MuIDSF  & 0.1/-0.2 \\
 \\
\bottomrule
\end{tabular}

\\
Channel: $e \mu$ \\
\begin{tabular}{r|p{.08\linewidth}}
\toprule
 Sys.  & FourTops \\
\toprule
JES  & 0.8/-2.4 \\
JER  & 0.1/-0.2 \\
BJET  & 1.0/-0.8 \\
JVF  & 4.0/-3.9 \\
SoftJets  & 0.0/-0.3 \\
MET  & 0.0/-0.3 \\
JEFF  & 0.0/-0.1 \\
EES  & 0.1/0.0 \\
EER  & 0.0/0.0 \\
MER  & 0.0/-0.5 \\
MES  & 0.0/-0.1 \\
MuTSF  & 1.4/-1.5 \\
ElTSF  & 0.5/-0.6 \\
ElRecoSF  & 0.9/-1.0 \\
ElIDSF  & 2.4/-2.5 \\
MuRecoSF  & 0.3/-0.4 \\
MuIDSF  & 0.8/-0.9 \\
 \\
\bottomrule
\end{tabular}

\\
Channel: $\mu \mu$ \\
\begin{tabular}{r|p{.08\linewidth}}
\toprule
 Sys.  & FourTops \\
\toprule
JES  & 0.0/-1.1 \\
JER  & 0.2/-0.3 \\
BJET  & 0.0/-0.5 \\
JVF  & 4.0/-3.9 \\
CellOutSoftJet  & 0.0/-0.3 \\
METPileUp  & 0.0/-0.3 \\
JEFF  & 0.2/-0.3 \\
EES  & 0.0/0.0 \\
EER  & 0.0/0.0 \\
MER  & 0.0/-0.4 \\
MES  & 0.0/-0.1 \\
MuTSF  & 2.8/-2.8 \\
ElTSF  & 0.0/-0.1 \\
ElRecoSF  & 0.0/-0.1 \\
ElIDSF  & 0.1/-0.2 \\
MuRecoSF  & 0.7/-0.8 \\
MuIDSF  & 1.5/-1.6 \\
 \\
\bottomrule
\end{tabular}

\\
\label{tab:systematics4t}
\end{centering}


\subsubsection{Systematic uncertainties on Monte Carlo backgrounds}
\begin{centering}
\tiny
Channel: $ee$ \\
\begin{tabular}{r|p{.08\linewidth}p{.08\linewidth}p{.08\linewidth}p{.08\linewidth}p{.08\linewidth}}
\toprule
 Sys.  & ttbar\_W & ttbar\_Z & ttbar\_WW & WZ\_ZZ & WWjj \\
\toprule
JES  & 9.3/-10.4 & 6.3/-7.7 & 3.2/-8.1 & 15.9/-14.4 & 6.0/-6.0 \\
JER  & 1.8/-1.9 & 0.6/-0.7 & 0.4/-0.5 & 1.7/-1.8 & 0.4/-0.5 \\
BJET  & 3.5/-3.8 & 1.7/-3.7 & 0.0/-2.1 & 0.0/0.0 & 0.0/-3.4 \\
JVF  & 2.6/-2.6 & 2.8/-2.7 & 2.9/-2.8 & 1.7/-1.7 & 1.7/-1.7 \\
CellOutSoftJet  & 0.9/0.0 & 0.0/0.0 & 0.0/-1.1 & 0.0/0.0 & 0.0/0.0 \\
METPileUp  & 0.9/0.0 & 0.0/0.0 & 0.0/-1.1 & 0.0/0.0 & 0.0/0.0 \\
JEFF  & 0.3/-0.4 & 0.3/-0.4 & 0.0/-0.1 & 1.1/-1.2 & 0.0/0.0 \\
EES  & 0.0/0.0 & 1.3/-0.8 & 1.3/0.0 & 0.1/0.0 & 0.0/0.0 \\
EER  & 0.6/-4.2 & 0.0/0.0 & 0.8/0.0 & 1.5/-1.2 & 0.0/0.0 \\
MER  & 0.0/0.0 & 0.0/0.0 & 0.0/0.0 & 0.0/0.0 & 0.0/0.0 \\
MES  & 0.0/0.0 & 0.0/0.0 & 0.0/-0.1 & 0.0/0.0 & 0.0/-0.1 \\
MuTSF  & 0.1/-0.2 & 0.0/-0.1 & 0.1/-0.2 & 0.0/0.0 & 0.0/0.0 \\
ElTSF  & 1.0/-1.1 & 1.2/-1.3 & 1.0/-1.1 & 1.1/-1.2 & 1.0/-1.1 \\
ElRecoSF  & 1.7/-1.8 & 2.1/-2.2 & 1.8/-1.9 & 2.1/-2.1 & 1.7/-1.7 \\
ElIDSF  & 4.7/-4.6 & 5.5/-5.4 & 4.7/-4.7 & 5.2/-5.1 & 4.5/-4.5 \\
MuRecoSF  & 0.0/-0.1 & 0.0/-0.1 & 0.0/0.0 & 0.0/0.0 & 0.0/0.0 \\
MuIDSF  & 0.0/-0.1 & 0.0/-0.1 & 0.0/-0.1 & 0.0/0.0 & 0.0/0.0 \\
 \\
\bottomrule
\end{tabular}

\\
Channel: $e \mu$ \\
\begin{tabular}{r|p{.08\linewidth}p{.08\linewidth}p{.08\linewidth}p{.08\linewidth}p{.08\linewidth}}
\toprule
 Sys.  & ttbar\_W & ttbar\_Z & ttbar\_WW & WZ\_ZZ & WWjj \\
\toprule
JES  & 8.1/-8.8 & 10.8/-8.3 & 9.7/-8.0 & 13.4/-12.6 & 6.3/-7.9 \\
JER  & 0.0/-0.1 & 1.0/-1.1 & 0.0/-0.1 & 1.7/-1.8 & 0.3/-0.4 \\
BJET  & 3.6/-2.6 & 1.9/-2.3 & 1.2/-1.5 & 0.0/0.0 & 0.0/0.0 \\
JVF  & 2.6/-2.5 & 2.7/-2.7 & 3.0/-2.9 & 1.6/-1.7 & 1.6/-1.6 \\
CellOutSoftJet  & 0.0/0.0 & 0.0/0.0 & 0.0/0.0 & 0.0/0.0 & 0.0/0.0 \\
METPileUp  & 0.0/0.0 & 0.0/0.0 & 0.0/0.0 & 0.0/0.0 & 0.0/0.0 \\
JEFF  & 0.0/-0.1 & 0.1/-0.2 & 0.1/-0.2 & 0.3/-0.4 & 0.0/-0.1 \\
EES  & 0.8/0.0 & 0.1/-0.2 & 0.6/0.0 & 0.0/0.0 & 0.9/-0.1 \\
EER  & 0.0/0.0 & 0.3/0.0 & 0.0/-0.1 & 0.0/-1.5 & 0.0/0.0 \\
MER  & 0.5/-1.1 & 0.7/-0.3 & 0.2/-0.1 & 3.0/0.0 & 0.8/0.0 \\
MES  & 0.0/-0.1 & 0.0/-0.1 & 0.1/-0.2 & 0.0/0.0 & 0.0/-0.1 \\
MuTSF  & 1.4/-1.5 & 1.6/-1.7 & 1.4/-1.5 & 1.8/-1.9 & 1.3/-1.4 \\
ElTSF  & 0.5/-0.6 & 0.6/-0.7 & 0.5/-0.6 & 0.6/-0.7 & 0.5/-0.6 \\
ElRecoSF  & 0.9/-0.9 & 1.1/-1.2 & 0.9/-1.0 & 1.0/-1.1 & 0.8/-0.9 \\
ElIDSF  & 2.3/-2.4 & 2.8/-2.8 & 2.3/-2.4 & 2.8/-2.9 & 2.2/-2.3 \\
MuRecoSF  & 0.3/-0.4 & 0.4/-0.5 & 0.3/-0.4 & 0.5/-0.6 & 0.3/-0.4 \\
MuIDSF  & 0.7/-0.8 & 0.9/-1.0 & 0.8/-0.9 & 1.0/-1.0 & 0.7/-0.8 \\
 \\
\bottomrule
\end{tabular}

\\
Channel: $\mu \mu$ \\
\begin{tabular}{r|p{.08\linewidth}p{.08\linewidth}p{.08\linewidth}p{.08\linewidth}p{.08\linewidth}}
\toprule
 Sys.  & ttbar\_W & ttbar\_Z & ttbar\_WW & WZ\_ZZ & WWjj \\
\toprule
JES  & 8.1/-9.9 & 5.4/-6.9 & 5.1/-4.0 & 6.0/-2.7 & 0.5/-10.9 \\
JER  & 0.6/-0.7 & 0.5/-0.6 & 0.1/-0.2 & 4.3/-4.4 & 2.6/-2.7 \\
BJET  & 4.1/-0.9 & 1.8/-3.0 & 0.5/-1.8 & 0.0/0.0 & 0.0/0.0 \\
JVF  & 2.6/-2.6 & 2.6/-2.6 & 2.9/-2.9 & 1.7/-1.7 & 1.6/-1.6 \\
SoftJets  & 0.2/0.0 & 0.0/-0.4 & 0.0/-0.5 & 0.0/0.0 & 0.0/0.0 \\
MET  & 0.2/0.0 & 0.0/0.0 & 0.0/0.0 & 0.0/0.0 & 0.0/0.0 \\
JEFF  & 0.0/-0.1 & 0.6/-0.7 & 0.1/-0.2 & 0.4/-0.5 & 0.0/0.0 \\
EES  & 0.0/-0.1 & 0.0/0.0 & 0.0/0.0 & 0.0/0.0 & 0.0/0.0 \\
EER  & 0.0/0.0 & 0.0/0.0 & 0.0/0.0 & 0.0/0.0 & 0.0/0.0 \\
MER  & 1.5/-0.4 & 0.8/-1.1 & 0.6/-0.4 & 4.7/-0.8 & 1.6/-1.5 \\
MES  & 0.1/-0.2 & 0.2/-0.3 & 0.0/-0.1 & 0.0/0.0 & 0.0/0.0 \\
MuTSF  & 2.6/-2.7 & 3.4/-3.4 & 2.7/-2.7 & 3.5/-3.5 & 2.4/-2.5 \\
ElTSF  & 0.0/-0.1 & 0.0/-0.1 & 0.0/-0.1 & 0.0/0.0 & 0.0/0.0 \\
ElRecoSF  & 0.0/-0.1 & 0.0/-0.1 & 0.0/-0.1 & 0.0/0.0 & 0.0/0.0 \\
ElIDSF  & 0.1/-0.2 & 0.0/-0.1 & 0.2/-0.3 & 0.0/0.0 & 0.0/0.0 \\
MuRecoSF  & 0.7/-0.8 & 0.9/-1.0 & 0.7/-0.8 & 1.0/-1.0 & 0.7/-0.8 \\
MuIDSF  & 1.4/-1.5 & 2.0/-2.0 & 1.5/-1.6 & 2.0/-2.0 & 1.4/-1.5 \\
 \\
\bottomrule
\end{tabular}

\\
\label{tab:systematicsBG}
\end{centering}

\subsection{Uncertainties on data-driven backgrounds}

\subsubsection{Charge flip background}
 The uncertainty on the overall scale of the charge-flip background in the signal regions is derived from a comparison of the charge-flip rate extracted by several methods.
The primary technique, which uses the likelihood method, is compared to the alternative Tag-and-Probe and Direct Extraction methods (see section~\ref{chap:smbkg}).
Based on these methods, an overall uncertainty of 12\% is attributed to the Charge-Flip background.

\subsubsection{Fake lepton background}

Uncertainties on the estimate of background coming from fake leptons are 50\% for the $ee$ channel, 40\% for the $e\mu$ channel, and 30\% for the $\mu\mu$ channel.
For the electron case, the uncertainty was estimated by using different {\it loose} electron definitions. For the muon case, it was estimated by changing the definition of the multijet enriched region, and using the number of jets, instead of the leading jet \pT{}, for the weight parametrization.







