
\section{Dilepton}


\subsection{Introduction}

While the branching ratio into the dilepton decay channel for top-quark pair production events
isn't as high as the lepton+jets channel, the presence of two high pt-leptons in the final state
makes this a powerful channel for precision measurements of top quark processes.

We here describe a measurement of the top-quark pair-production cross-section using the
dilepton final state, which is characterized by two high-pt, opposite-sign leptons,
significant missing energy due to the presence of neutrinos from the decay of W-bosons,
and two $b$-jets from the decay of the top quarks.


%% Top-quark production in dilepton final states has been studied using
%% proton-antiproton collisions at
%% $\sqrt{s}=1.96~\TeV$~\cite{Aaltonen:2010bs,Abazov:2009ae} and
%% Large Hadron Collider (LHC)
%% measurements of the production cross section in proton-proton
%% collisions at $\sqrt{s}=7~\TeV$ in the same final state have recently been
%% reported~\cite{Chatrchyan:2011nb,ATL-CONF-2011-034}.
%% A measurement is presented of the \ttbar\ production cross section
%% using the dilepton channel, characterized by two opposite-sign
%% leptons, unbalanced transverse momentum indicating the presence of
%% neutrinos from the $W$-boson decays, and two $b$-quark jets. This
%% result uses twenty times more data than the previous ATLAS measurement in the same final state,
%% reported in Ref.~\cite{ATL-CONF-2011-034}.

The high efficiency of lepton selection and the relatively small contribution of backgrounds
to the dilepton final state combined with a requirement on a $b$-jet that further reduces
background gives this channel a strong signal-to-background ratio.
The dilepton decay topology is divided into three sub-channels based on the flavor of the
final-state leptons.
The $ee$ channel contains two high-pt, isolated leptons identified using 
the inner detector and the electromagnetic calorimeter, the $\mu\mu$ channel contains two
high-pt, isolated muons which have been identified using the inner detector as well as the muon
spectrometer, and the $e \mu$ final state contains an electron and a muon.
These channels are mutually exclusive to facilitate their statistical combination.


%% The \ttbar~dilepton final states can be selected with a good
%% signal-to-background ratio using simple kinematic requirements.  With
%% the additional requirement of the
%% presence of a jet consistent with a $b$~quark
%% (`$b$-tag'), the signal-to-background ratio can be further improved.
%% Cross-section measurements with and without the $b$-tag requirement
%% are reported here.
%%  Leptons are either well-identified electron or
%% muon candidates that are selected using the full detector or, to
%% reduce losses from lepton identification inefficiencies, isolated
%% tracks.  The well-identified electrons or muons are called
%% `identified leptons', and the isolated tracks are referred to as
%% `track leptons'. The term `lepton' is used to refer to identified
%% leptons and track leptons collectively. Events with one identified
%% lepton and one track lepton are called `lepton+track' events.
%%   Each dilepton channel is
%% exclusive, i.e. has no overlap with the other channels. Channels
%% with tau leptons are not explicitly reconstructed, but reconstructed
%% leptons can arise from leptonic tau decays and a track lepton can
%% arise from hadronic tau decay modes as well.  The analysis with the
%% $b$-tag requirement uses only identified leptons.

The analysis uses collision data with a center-of-mass energy of
$\sqrt{s}$ = 7~\TeV\ from 2011, with an integrated luminosity
of \lumitotpm~\cite{lumiPub, lumi}.
The main backgrounds considered in this analysis come from
$\Z$+jets events, events with a single top quark,
diboson events, and events with misidentified leptons originating from hadronic activity.
Background contributions from
$\Zg\rightarrow ee$+jets, $\Zg\rightarrow\mu\mu$+jets and events with
misidentified leptons are estimated using data-driven techniques, and all
other background estimations are obtained using Monte Carlo simulation.


%% The measured cross section takes into account the $\ttbar$\ signal
%% acceptance and the expected background contributions from
%% $\Zg$+jets, single top quarks, $WW$, $WZ$, and $ZZ$ events, and
%% events with misidentified leptons (primarily $W+$jets events).
%% Background contributions from
%% $\Zg\rightarrow ee$+jets, $\Zg\rightarrow\mu\mu$+jets and events with
%% misidentified leptons are evaluated
%% directly from the data. All other background contributions are
%% evaluated using Monte Carlo (MC) simulation samples.


\subsection{Monte Carlo Simulation}
\label{s:mc}


Monte Carlo simulation is used to estimate the production rate and kinematic distributions of the $\ttbar$ signal as well as several important background, 
and in addition it is used to evaluate the size of many important sources of systematic uncertianty.
All Monte Carlo samples used in this analysis were generated using \GEANT~\cite{geant4}\
to simulate the ATLAS detector.


%% Monte Carlo simulation samples are used to calculate the $\ttbar$\
%% acceptance and to evaluate the background contributions from single
%% top quarks, $WW$, $WZ$, and $ZZ$ events, and $\Zg
%% \rightarrow\tau\tau$+jets.
%% All MC samples are processed with the
%% \GEANT~\cite{geant4}\ simulation of the ATLAS detector~\cite{atlsim}\
%% and events are passed through the same analysis chain as the data.

Events containing top quarks, including the $\ttbar$ signal and the single-top background,
were generated using \MCatNLO\ generator~\cite{mcatnlo1,Frixione:2003ei,Frixione:2005vw}\
with the CTEQ6.6~\cite{cteq66}\ parton distribution function (PDF)
set and a top-quark mass of 172.5~\GeV.
However, the reference for the overall $\ttbar$ cross-section was calculated using
{\sc Hathor}\cite{Aliev:2010}, which employs an NNLO perturbative QCD calculation.


%% The generation of \ttbar{} and single top-quark events uses the
%% \MCatNLO\ generator~\cite{mcatnlo1,Frixione:2003ei,Frixione:2005vw}\
%% with the CTEQ6.6~\cite{cteq66}\ parton distribution function (PDF)
%% set and a top-quark mass of 172.5~\GeV. Expected $\ttbar$\ yields
%% are calculated with a cross section normalized to the prediction of
%% {\sc Hathor}\cite{Aliev:2010}, which employs an NNLO perturbative QCD calculation.
%% Single top-quark production with \MCatNLO\ includes the $s$, $t$ and
%% $Wt$ channels and the diagram-removal scheme~\cite{diagrem}\  is
%% used to reduce overlap with the $\ttbar$\ final state.

Drell-Yan events are simulated using the {\sc Alpgen} generator using
CTEQ6L1 \cite{cteq6l}\ to model the parton distribution function (pdf),
and the overall cross-section is scaled to the NNLO prediction, which
requires a scale factor (k-factor) of 1.25.
Diboson events are also modeled using the {\sc Alpgen}\ generator
and the overall cross-sections of the three categories of diboson events
were each individually scaled to match their predictions from NLO QCD,
which were calculated using the MCFM program~\cite{Campbell:1999ah}.

%% Drell-Yan events ($\Zg$+jets) are modeled with the {\sc Alpgen}
%% generator, using the MLM matching scheme~\cite{alpgen}\ and the
%% CTEQ6L1 \cite{cteq6l}\ PDF set. The $\Zg$+jets samples, including
%% both light and heavy flavor jets, are normalized to NNLO with a
%% $K$-factor of 1.25. %~\cite{CSCbook}.
%% In the $\Zg\rightarrow ee$ and $\mu\mu$ decay channels, the
%% background from $\Zg$+jets is evaluated using a data-driven
%% technique that normalizes the MC expectation to the data observation
%% near the $Z$ pole. Background contributions from the $W$+jets final
%% states come primarily from events where the $W$\ boson decays
%% leptonically and the second lepton candidate is a misidentified jet
%% or a heavy-flavor decay. Backgrounds from $W$+jets %and multijet
%% events are evaluated from the data.

All simulated event uses the {\sc Herwig} to model the parton shower ~\cite{herwig1,Corcella:2002jc}
as well as {\sc Jimmy} to simulate the effect of the underlying event model~\cite{jimmy}.
In addition, all events include additional background interactions, generated using Pythia,
to simulate the effect of in-time pile-up at the LHC.
Individual simulation events have been re-weighed so the overall distribution of
additional events due to pile-up matches the distribution measured in data.

%% All MC simulated events are hadronized using the {\sc Herwig}\ shower model~\cite{herwig1,Corcella:2002jc}\ supplemented by the
%% {\sc Jimmy}\ underlying event model~\cite{jimmy}.
%% Both hadronization programs are tuned to ATLAS data using the ATLAS MC10 tune~\cite{ATLAS:1303025}.
%% Diboson events are modeled using the {\sc Alpgen}\ generator normalized with
%% $K$-factors of 1.26 ($WW$), 1.28 ($WZ$) and 1.30 ($ZZ$) to match the total cross section from NLO QCD
%% predictions using calculations with the MCFM program~\cite{Campbell:1999ah}.

%% All Monte Carlo samples are generated taking into account that
%% multiple $pp$ interactions can occur in the same LHC bunch crossing
%% within a given event (`pile-up').  The MC events are re-weighted so
%% that the distribution of interactions per crossing in the MC matches
%% that observed in the data.  The average number of interactions per
%% crossing is 5.6 in this data set.


\subsection{Object selection}
\label{s:object}

The criteria for identifiying physical objects is consistent across the three dilepton sub-channels.
%Both electrons and muons are triggered using a single-lepton trigger

Electrons are identified as clusters in the electromagnetic calorimeter
that are matched to charged tracks in the inner-detector.
Electrons used in this analysis must meet the ``tight'' identification criteria,
which places strict requirements on the quality of the inner detector track and the
shape of the shower in the calorimeter.
Electrons are required to have $\pt>25\GeV$, are required to be within the range of the
optimal ATLAS inner detector, namely having $|\eta_{\text cl}|<2.47$ excluding the
transition region defined by $1.37<|\eta_{\text cl}|<1.52$.
Further, electrons are required to be isolated the electromagnetic calorimeter
by requiring that the transverse energy not associated with the electron within 
a radius defined by $\Delta R\equiv\sqrt{\Delta\eta^2 + \Delta\phi^2}$ must be less than $3.5\GeV$.

%% Lepton isolation requirements reduce backgrounds from misidentified
%% jets and suppress the selection of leptons from heavy-flavor decays.
%% For electron candidates, the transverse energy ($\ET$) deposited in
%% the calorimeter not associated to the electron is summed in a cone
%%  of radius $\Delta R = 0.2$, where
%% electron and is required to be less than $3.5\GeV$.


%% Leptons are required to be isolated and have high transverse
%% momentum, $\pt$, consistent with originating from $W$-boson
%% decay, with $\pt$ thresholds chosen to ensure events are triggered
%% with high efficiency.

%% Electron candidates are reconstructed from energy deposits
%% (clusters) in the EM calorimeter, which are then associated to
%% reconstructed tracks of charged particles in the inner detector.
%% Stringent quality requirements on the conditions of the EM
%% calorimeter at the time of data taking are applied to ensure a well measured reconstructed
%% energy. A `tight' selection~\cite{ElectronPerformance} using calorimeter,
%% tracking and combined variables, is employed to provide good
%% separation between the signal electrons and background.  Electron
%% candidates are additionally required to have $\pt>25\GeV$ and
%% $|\eta_{\text cl}|<2.47$, excluding electrons from the transition
%% region between the barrel and endcap calorimeters defined by
%% $1.37<|\eta_{\text cl}|<1.52$. The variable $\eta_{\text cl}$\ is
%% the pseudorapidity of the energy cluster associated with
%% the candidate.

Muons are required to contain a spectrometer track that matches a track in the inner detector.
They must have $\pt>20\GeV$ and $|\eta|<2.5$.
They are required to be energetically isolated both in the muon spectrometer and in the
inner detector.
The sum of the momenta of tracks having within $\Delta R = 0.3$\ and having $\pt>1$~GeV
must be no more than 4~GeV and the sum of transverse momentum within
a cone of $\Delta R = 0.3$\ must also be less than 4~GeV.
In addition, muons must be separated by at least $\Delta R > 0.4$\ from
selected jets wtih $\pt>20$~GeV, and potential muon pairs coming from
cosmic rays are eliminated by rejecting back-to-back muons with $|d_0|~>~0.5~$mm.

%% Muon candidate reconstruction is begun by searching for track
%% segments in
%%  layers of the muon chambers.
%% These segments are  combined starting from the
%% outermost layer, fitted to account for material effects, and matched
%% with tracks found in the inner detector.
%% The candidates are refitted using the complete track information from
%% both detector systems%~\cite{CSCbook}
%% , and required to satisfy $\pt>20\GeV$ and
%% $|\eta|<2.5$.

%For muon candidates, both the
%corresponding calorimeter isolation energy and the analogous track
%isolation, the sum of the track transverse momenta for tracks with
%$\pt>1$~GeV and in a cone $\Delta R = 0.3$\ centered on the lepton
%candidate, must be less than 4~GeV.

%% Additionally, muon candidates
%% must have a distance $\Delta R > 0.4$\ from any jet with
%% $\pt>20$~GeV, further suppressing muon candidates from heavy flavor
%% decays.
%% Muon candidates arising from cosmic rays are rejected by removing
%% candidate pairs that are back-to-back in the $r-\phi$ plane and
%% with transverse impact parameters relative to the beam axis
%% $|d_0|~>~0.5~$mm.

% Ignore Track Leptons

%% Track-lepton (TL) candidates are defined by an ID track with $\pt >
%% 25\GeV$ and a series of quality cuts optimized for high efficiency
%% and a low rate of misidentification. The track must have at least
%% six  SCT hits  and at least one hit in the innermost pixel layer. It
%% also must have $|d_0| < 0.2$~mm and the uncertainty on the momentum
%% measurement must be less than 20\%. The track has to be isolated
%% from other nearby tracks, following the track isolation defined above, in this case using
%%  tracks with $\pt>0.5\GeV$.  The summed momentum cut is set to 2~GeV.

Jets are reconstructed from energy clusters in the calorimeter using the {\it \AKT} algorithm~\cite{antikt}
with a radius parameter $R = 0.4$.
Jets are calibrated to the hadronic energy scale using $\eta$ and $\phi$ dependent corrections.
Because both jets and electrons are seeded by energy clusters in the calorimeter,
reconstructed jets within $\Delta R=0.2$\ of a selected electron are removed to
prevent double counting of physical objects.
Finally, jets are required to have $\pt>25$~GeV and $|\eta|<2.5$.

%% %of adjacent calorimeter cells.
%% These jets are then calibrated to
%% %by first
%% %correcting the jet energy using the scale established for
%% %electromagnetic objects and then performing a further correction to
%% the hadronic energy scale using $\pt$\ and $\eta$\ dependent
%% correction factors~\cite{jetcor}.
%% %Jets are removed if they are within $\Delta R=0.2$\ of a well-identified
%% %electron candidate. In lepton+track events, jets are removed if they are within $\Delta R=0.2$\ of a TL.
%% Jets are removed if they are within $\Delta R=0.2$\ of a well-identified
%% electron candidate or a TL.
%% The jets used in the analysis are required to have $\pt>25$~GeV
%% and $|\eta|<2.5$.

Selected jet candidates can be further classified as originating from a
$b$-quark using an algorithm based on a likelihood ratio.
The likelihood discriminates bewteen $b$-jets and jets originating from
light quarks using a number of features, including the impact parameter
of the jet, the decay length significance associated with the secondary
vertex, the invariant mass of the tracks associated with the jet, 
and the ratio of the sum of energies of the tracks to the total
energy of the jet itself.
The classifier is designed to have an efficiency of $\approx 80\%$\
when applied to $b$-jets arising from $\ttbar$ events.


%% Jets are identified as $b$-quark candidates (`$b$-tagged') by an
%% algorithm that forms a likelihood ratio of $b$- and light-quark jet
%% hypotheses using the following discriminating variables: the signed
%% impact parameter significance of well measured tracks associated
%% with a given jet, the decay length significance associated with a
%% reconstructed secondary vertex, the invariant mass of all tracks
%% associated to the secondary vertex, the ratio of the sum of the
%% energies of the tracks associated with the secondary vertex to the
%% sum of the energies of all tracks in the jet assuming a pion
%% hypothesis, and the number of
%% two-track vertices that can be formed at the secondary
%% vertex~\cite{AdvBtag}. The cut on the combined likelihood ratio has
%% been chosen such that a $b$-tagging efficiency of $\approx 80\%$\
%% per $b$-jet in \ttbar~candidate events is achieved.
%and light quark as well as gluon jet rejection of order ten are achieved.

The missing transverse energy is formed by using the transverse 
momenta of jets, the calibrated transverse momenta of the cells of electron
candidates, muons, and calorimeter clusters not associated with a
reconstructed object.


%% The missing transverse momentum is formed from the negative vector
%% sum of transverse momenta of all jets with $\pt>20$~GeV and $|\eta|
%% < 4.5$~\cite{METPaper}. The contribution from cells associated with
%% electron candidates is replaced by the candidates' calibrated
%% transverse energy. The contribution from all muon candidates  and
%% calorimeter clusters (including those not belonging to a reconstructed
%% object) is also included.
%% The symbol $\MET$ is used to denote the magnitude of the
%% missing transverse momentum.


%
% HERE
%

\subsection{Event selection}
\label{s:event}

The analysis requires collision data selected by an inclusive single
%lepton trigger ($e$ or $\mu$)
electron or muon trigger with offline-reconstructed
candidates satisfying $\pt>25\GeV$ for electrons, and $\pt>20\GeV$ for muons, to ensure a constant trigger efficiency.
To ensure that the event was triggered by the lepton candidates used
in the analysis,  one of the identified leptons and the triggered
lepton are required to match within $\Delta R < 0.15$.

Events are required to have a primary interaction vertex with at
least five tracks with $\pt>400~\MeV$.  The event is discarded if
any jet with $\pt>20$~GeV fails quality cuts designed to reject jets
arising from calorimeter noise or activity inconsistent with the
bunch-crossing time~\cite{jetcor}. If an electron candidate and a
muon candidate share a track, the event is also discarded.

The selection of events in the signal region consists of a series of kinematic requirements on
the reconstructed objects.
The requirements on $\MET$, the lepton-lepton invariant
mass ($m_{\ell\ell}$), and the scalar $\pt$ sum of all selected jets and leptons ($\HT$) are optimized
to minimize the expected total uncertainty on the cross-section measurement.
The resulting event selection, referred to as the `non-$b$-tag' selection, is listed below.

\begin{itemize}

\item Events must have exactly two oppositely-charged identified-lepton candidates ($ee$, $\mu\mu$,
  $e\mu$), satisfying the selection criteria of Section~4, {\em or} if only one identified-lepton candidate is found, the event is retained if a track-lepton candidate is present, with opposite charge to the identified lepton, forming a lepton+track event ($e$TL or $\mu$TL).

\item Events must have at least two jets with $\pT>25\GeV$ and $|\eta|<2.5$.%, no $b$-tagged jets are required.

\item Events in the $ee$, $\mu\mu$ $e$TL and $\mu$TL channels are required to have $m_{\ell\ell} > 15\GeV$ in order to
reject backgrounds from vector-meson decays. The requirement also helps to suppress backgrounds in these channels from $b$-quark production.

\item Events in the $ee$ and $\mu\mu$ channels must satisfy $\MET>60\GeV$ and $|m_{\ell\ell} - m_Z| > 10\GeV$,
to suppress backgrounds from $\Zg$+jets and multijets.

\item
Events in the $e\mu$ channel are required to satisfy $\HT>130\GeV$. No \MET\ or $m_{\ell\ell}$\ cuts are applied.
%In this case, remaining background from $\Zg$+jets production
%is suppressed by requiring $\HT>130~\GeV$.

\item The lepton+track event candidates must have  $\MET>45\GeV$, \HT\ (including the
track lepton) $>150\GeV$, and $|m_{\ell\ell} - m_Z| > 10\GeV$.

\end{itemize}

A parallel selection with the additional requirement of at least one
$b$-tagged jet is made. Because of the enhanced background rejection
afforded by the $b$-tag requirement, the selection is further
optimized, resulting in an $\MET$\ requirement for $ee$ and $\mu\mu$
events that is relaxed to $\MET>40\GeV$, while the $\HT$\
requirement for $e\mu$ events remains the same as for the
non-$b$-tag selection, i.e. $\HT> 130\GeV$.  We refer to the
analysis that requires at least one $b$-tagged jet as the `$b$-tag
analysis', and the events selected therein as the `$b$-tagged
sample'.  The subset of the $b$-tagged sample with $40\GeV<
\MET<60\GeV$ is referred to as the `exclusive $b$-tagged sample' and
has no overlap with the non-$b$-tag sample.

The acceptance times the branching fraction of $\ttbar$ to
dileptons, for the selection described above, is 0.96\% for the $ee
+ \mu\mu +e\mu$ channels without $b$-tagging, 0.11\% for the
exclusive $b$-tagged sample, and 0.19\% for $e$TL + $\mu$TL
channels.


\subsection{Background evaluation}
\label{s:backgrounds}

The $\ttbar$ event selection rejects $\Zg$+jets
events with $ee$ and $\mu\mu$ invariant mass below $15 \GeV$, or within $10 \GeV$ of the $Z$-boson mass. However, $\Zg$+jets events with $ee$ or $\mu\mu$  invariant mass outside of these regions can enter the signal sample when there is large $\met$, typically from mismeasurement.
These events are difficult to properly model in simulations
due to uncertainties on the non-Gaussian tails of the $\met$\ distribution,
on the cross section for $Z$~boson production with multiple jets, and on the
lepton energy resolution.

To evaluate the $\Zg$+jets background in dielectron and dimuon events
($Z \to \tau\tau$ is considered below), the MC prediction for the number of events in the signal region is normalized to the data using the number of
$\Zg$+jets events measured in a control region~\cite{ATL-CONF-2011-034}.
%orthogonal to the \ttbar\ dilepton signal region.
The control region is formed by
events with the same jet requirements as the signal region, but with
 $m_{\ell \ell}$ within 10 GeV of the $Z$-boson mass, and a $\MET$ cut of
$\met>45~\GeV$ for the lepton+track candidates and $\met>30~\GeV$
for the others. Contamination in the control region from other
physics processes (signal and other background processes considered
for the analysis) is subtracted according to MC predictions.
The ratio of data events to MC expectation in the control region
provides a scale factor that is used to correct the MC prediction
for $\Zg$+jets events in the signal region.


Other backgrounds mainly come from $W$+jets, \ttbar\, lepton+jets,
and single top-quark production with fake leptons. The term `fake
lepton' is used to refer to both misidentified and non-prompt lepton
candidates, the latter category arising from hadron decays in
flight. The yield of events with fake identified leptons is
evaluated from the data using a matrix method~\cite{top2010}. In
addition to the standard lepton selection requirements, a selection
with a looser isolation requirement is defined.  Dilepton events are
selected using the loose isolation requirement and events are
categorized according to whether each lepton passes the standard
selection or the loose selection but not the standard selection.
There are four such categories for the two leptons: loose-loose,
loose-standard, standard-loose, and standard-standard.  Each of the
four categories is related to the number of events with two `real'
(prompt) leptons, two fake leptons, or one of each, through a set of
linear equations with coefficients given by the products of
probabilities for real or fake lepton satisfying the loose selection
to also satisfy the standard selection.  These linear expressions
form a matrix that is inverted in order to extract the real and fake
lepton content of the observed dilepton event sample.
The probability for real leptons is measured as a function of jet
multiplicity using data samples of
$\Zee$ and $\Zmm$ events.
The corresponding probability for
fake leptons is measured in a data sample dominated by dijet production,
with events containing one lepton candidate passing the looser isolation cuts and having $\met <20 \GeV$.
Contributions from real leptons due to $W$+jets final states are
subtracted using simulated events.

For lepton+track events the largest background is from events with
fake leptons, dominated by fake track leptons. The
probability of a jet being reconstructed as a track lepton is
determined from a $\gamma$+jets data sample selected with photon
triggers. The fake probability is applied to a second sample
enriched in $W$+jets events with exactly one identified lepton and
no track leptons, but using the same kinematic cuts as for the
signal sample. In this second sample the fake probabilities are
summed for each jet in each event and the fake track-lepton
contribution is calculated as a function of the number of jets.

The contributions from other electroweak background processes with
two real leptons,  such as single top quarks, $Z \to \tau\tau$,
$WW$, $ZZ$ and $WZ$ production are determined from Monte Carlo
simulations.   The expected numbers of background events are given
in Table~\ref{t:signal}.  The absence of $\Zg$+jets background in
the $e\mu$ channel, coupled with the larger branching fraction for
the $e\mu$ signal, allows relatively loose selection criteria to be
used in this channel. %The $b$-tagged $e\mu$ events require the same

The background contributions for the $b$-tag analysis are determined
using the same techniques described above, with the additional
requirement of a $b$-tagged jet.

The modeled acceptances, efficiencies and data-driven background
evaluation methods are validated by comparing predictions from Monte
Carlo simulations with data in control regions with kinematics
similar to the signal region but dominated by backgrounds. In
particular, the \met, $m_{\ell\ell}$ and jet multiplicity
distributions are studied in a sample of $Z$-boson candidates,
defined by requiring $|m_{\ell\ell} - m_Z| < 10 \GeV$ and $\met
<$~60 GeV. The predictions from MC simulations in the control
regions are in reasonable agreement with data, although small
discrepancies exist in regions that do not affect the $\ttbar$
cross-section measurement.


\newcommand{\DYZeeNJetsTwoJet}{$4.0^{+2.5}_{-1.2}$}
\newcommand{\DYZmmNJetsTwoJet}{$14.4^{+5.4}_{-4.2}$}
\newcommand{\ZtteeNJetsTwoJet}{4.9 $\pm$ 2.6}
\newcommand{\ZttmmNJetsTwoJet}{11.0 $\pm$ 5.0}
\newcommand{\ZttemNJetsTwoJet}{43 $\pm$ 16}
\newcommand{\FakeWeeNJetsTwoJet}{4.0 $\pm$ 5.0}
\newcommand{\FakeWmmNJetsTwoJet}{6.3 $\pm$ 4.1}
\newcommand{\FakeWemNJetsTwoJet}{44 $\pm$ 24}
\newcommand{\singletopeeNJetsTwoJet}{6.4$^{+1.2}_{-1.1}$}
\newcommand{\singletopmmNJetsTwoJet}{16.0$^{+1.9}_{-2.2}$}
\newcommand{\singletopemNJetsTwoJet}{41.1 $\pm$ 5.5}
\newcommand{\dibosoneeNJetsTwoJet}{5.9 $\pm$ 1.1}
\newcommand{\dibosonmmNJetsTwoJet}{8.7$^{+1.2}_{-1.5}$}
\newcommand{\dibosonemNJetsTwoJet}{32.9 $\pm$ 4.9}

\newcommand{\TotalNonttbaree}{25.2 $\pm$ 6.4}
\newcommand{\TotalNonttbarmm}{56.5 $\pm$ 9.4}
\newcommand{\TotalNonttbarem}{161 $\pm$ 34}

\newcommand{\ttbareeNJetsTwoJet}{124 $\pm$ 17}
\newcommand{\ttbarmmNJetsTwoJet}{$241^{+15}_{-18}$}
\newcommand{\ttbaremNJetsTwoJet}{746 $\pm$ 42}

\newcommand{\TotalExpectedee}{149 $\pm$ 18}
\newcommand{\TotalExpectedmm}{$298^{+17}_{-20}$}
\newcommand{\TotalExpectedem}{907 $\pm$ 54}

\newcommand{\DataeeNJetsTwoJet}{165}
\newcommand{\DatammNJetsTwoJet}{301}
\newcommand{\DataemNJetsTwoJet}{963}

%Lepton+track
\newcommand{\DYZeTLNJetsTwoJet}{$24.3^{+10.7}_{-9.4}$}
\newcommand{\DYZmTLNJetsTwoJet}{$22.0^{+5.3}_{-5.8}$}
\newcommand{\ZtteTLNJetsTwoJet}{$17.0^{+8.4}_{-7.6}$}
\newcommand{\ZttmTLNJetsTwoJet}{25 $\pm$ 11}
\newcommand{\FakeWeTLNJetsTwoJet}{74 $\pm$ 15}
\newcommand{\FakeWmTLNJetsTwoJet}{85 $\pm$ 17}
\newcommand{\singletopeTLNJetsTwoJet}{$5.7^{+1.0}_{-0.9}$}
\newcommand{\singletopmTLNJetsTwoJet}{$6.3^{+0.8}_{-1.1}$}
\newcommand{\dibosoneTLNJetsTwoJet}{$5.9^{+0.9}_{-0.8}$}
\newcommand{\dibosonmTLNJetsTwoJet}{$4.8^{+0.6}_{-0.7}$}

\newcommand{\TotalNonttbareTL}{$126^{+20}_{-19}$}
\newcommand{\TotalNonttbarmTL}{142 $\pm$ 21}

\newcommand{\ttbareTLNJetsTwoJet}{$112~^{+16}_{-18}$}
\newcommand{\ttbarmTLNJetsTwoJet}{$110~^{+17}_{-16}$}

\newcommand{\TotalExpectedeTL}{239 $\pm$ 26}
\newcommand{\TotalExpectedmTL}{253 $\pm$ 27}

\newcommand{\DataeTLNJetsTwoJet}{236}
\newcommand{\DatamTLNJetsTwoJet}{255}

%b-tag
\newcommand{\DYZeeNJetsTwoJetb}{$9.8^{+1.7}_{-1.3}$}
\newcommand{\DYZmmNJetsTwoJetb}{$20.3^{+1.8}_{-2.8}$}
\newcommand{\ZtteeNJetsTwoJetb}{$1.8^{+1.1}_{-1.2}$}
\newcommand{\ZttmmNJetsTwoJetb}{7.6$^{+3.3}_{-3.6}$}
\newcommand{\ZttemNJetsTwoJetb}{9.5$^{+4.2}_{-3.9}$}
\newcommand{\FakeWeeNJetsTwoJetb}{7.5 $\pm$ 6.5}
\newcommand{\FakeWmmNJetsTwoJetb}{4.9 $\pm$ 3.1}
\newcommand{\FakeWemNJetsTwoJetb}{20 $\pm$ 13}
\newcommand{\singletopeeNJetsTwoJetb}{7.3$^{+1.3}_{-1.1}$}
\newcommand{\singletopmmNJetsTwoJetb}{16.2$^{+2.2}_{-2.3}$}
\newcommand{\singletopemNJetsTwoJetb}{33.5$^{+4.8}_{-4.7}$}
\newcommand{\dibosoneeNJetsTwoJetb}{2.2 $\pm$ 0.7}
\newcommand{\dibosonmmNJetsTwoJetb}{2.6$^{+0.9}_{-0.6}$}
\newcommand{\dibosonemNJetsTwoJetb}{8.8$^{+1.7}_{-1.6}$}

\newcommand{\TotalNonttbareeb}{28.6 $\pm$ 6.9}
\newcommand{\TotalNonttbarmmb}{51.6$^{+5.6}_{-5.9}$}
\newcommand{\TotalNonttbaremb}{71.6 $\pm$ 14.1}

\newcommand{\ttbareeNJetsTwoJetb}{159$^{+17}_{-21}$}
\newcommand{\ttbarmmNJetsTwoJetb}{$304^{+26}_{-35}$}
\newcommand{\ttbaremNJetsTwoJetb}{675$^{+57}_{-75}$}

\newcommand{\TotalExpectedeeb}{188$^{+18}_{-22}$}
\newcommand{\TotalExpectedmmb}{$356^{+27}_{-35}$}
\newcommand{\TotalExpectedemb}{746$^{+59}_{-76}$}

\newcommand{\DataeeNJetsTwoJetb}{201}
\newcommand{\DatammNJetsTwoJetb}{365}
\newcommand{\DataemNJetsTwoJetb}{834}

\begin{table*}[htb]
\begin{footnotesize}
\begin{adjustwidth}{-6em}{-6em}
\centering
\begin{tabular}{|l|c|c|c|c|c||c|c|c|} \hline
                          & $ee$                    & $\mu\mu$                & $e\mu$                  & $e$TL                & $\mu$TL      & $b$-tag $ee$         & $b$-tag $\mu\mu$      & $b$-tag $e\mu$          \\ [0.2ex] \hline
                          &                         &                         &                         &                      &              &                      &                       &                         \\ [-1.9ex]
                          $\Zg$+~jets              & \DYZeeNJetsTwoJet       & \DYZmmNJetsTwoJet       & -                       & \DYZeTLNJetsTwoJet  & \DYZmTLNJetsTwoJet         & \DYZeeNJetsTwoJetb     & \DYZmmNJetsTwoJetb    & $-$                             \\ [0.3ex]
$\Zg \to \tau\tau$+jets  & \ZtteeNJetsTwoJet       & \ZttmmNJetsTwoJet       & \ZttemNJetsTwoJet       & \ZtteTLNJetsTwoJet  & \ZttmTLNJetsTwoJet                & \ZtteeNJetsTwoJetb            & \ZttmmNJetsTwoJetb     &  \ZttemNJetsTwoJetb           \\ [0.3ex]
 Fake leptons & \FakeWeeNJetsTwoJet     & \FakeWmmNJetsTwoJet     & \FakeWemNJetsTwoJet     & \FakeWeTLNJetsTwoJet          & \FakeWmTLNJetsTwoJet                   & \FakeWeeNJetsTwoJetb     & \FakeWmmNJetsTwoJetb     & \FakeWemNJetsTwoJetb           \\ [0.3ex]
Single top quark              & \singletopeeNJetsTwoJet & \singletopmmNJetsTwoJet & \singletopemNJetsTwoJet & \singletopeTLNJetsTwoJet         & \singletopmTLNJetsTwoJet         & \singletopeeNJetsTwoJetb & \singletopmmNJetsTwoJetb & \singletopemNJetsTwoJetb       \\ [0.3ex]
Diboson                  & \dibosoneeNJetsTwoJet   & \dibosonmmNJetsTwoJet   & \dibosonemNJetsTwoJet   & \dibosoneTLNJetsTwoJet          & \dibosonmTLNJetsTwoJet         & \dibosoneeNJetsTwoJetb   & \dibosonmmNJetsTwoJetb   & \dibosonemNJetsTwoJetb        \\ [0.3ex] \hline
                          &                         &                         &                         &                      &              &                      &                       &                         \\ [-1.9ex]
 Total background         & \TotalNonttbaree        & \TotalNonttbarmm        & \TotalNonttbarem        & \TotalNonttbareTL     & \TotalNonttbarmTL       & \TotalNonttbareeb        & \TotalNonttbarmmb        & \TotalNonttbaremb           \\ [0.3ex]
Predicted \ttbar\       & \ttbareeNJetsTwoJet     & \ttbarmmNJetsTwoJet     & \ttbaremNJetsTwoJet     & \ttbareTLNJetsTwoJet       & \ttbarmTLNJetsTwoJet         & \ttbareeNJetsTwoJetb     & \ttbarmmNJetsTwoJetb     & \ttbaremNJetsTwoJetb \\ [0.3ex] \hline
                          &                         &                         &                         &                      &              &                      &                       &                         \\ [-1.9ex]
Total                 & \TotalExpectedee        & \TotalExpectedmm        & \TotalExpectedem        & \TotalExpectedeTL & \TotalExpectedmTL                 & \TotalExpectedeeb        & \TotalExpectedmmb       & \TotalExpectedemb \\ [0.3ex] \hline \hline
                          &                         &                         &                         &                      &              &                      &                       &                         \\ [-1.9ex]
Observed               & \DataeeNJetsTwoJet      & \DatammNJetsTwoJet      & \DataemNJetsTwoJet      & \DataeTLNJetsTwoJet                   & \DatamTLNJetsTwoJet                                    & \DataeeNJetsTwoJetb      & \DatammNJetsTwoJetb      & \DataemNJetsTwoJetb                      \\ \hline
\end{tabular}
\end{adjustwidth}
\end{footnotesize}
\caption {Breakdown of the expected {\ttbar} signal and background
events in the signal region compared to the observed event yields,
for each of the dilepton channels. All systematic uncertainties are
included, and correlations between different background sources are
taken into account, when calculating the total background
uncertainty. The largest contribution to the line labeled 'Fake
leptons' comes from $W+$jets events.} \label{t:signal}
\end{table*}




\subsection{Systematic uncertainties}
\label{s:objsyst}
A summary of the systematic uncertainties on the measured $\ttbar$ production cross section is given in Table~\ref{tab:syssum}.

Lepton trigger and identified-lepton and track-lepton
reconstruction and selection efficiencies are assessed using
$Z\rightarrow ee$ and $Z\rightarrow\mu\mu$ events in the same data
sample as used for the $\ttbar$ analyses.
Scale factors are evaluated by comparing these efficiencies with those determined with simulated $Z$-boson events. The scale factors are applied to MC samples when calculating
acceptances to account for any differences between predicted and observed efficiencies.  Systematic uncertainties on these scale factors are evaluated by varying the selection of events used in the efficiency measurements and by checking the stability of the measurements over the course of the data-taking period.

The modeling of lepton momentum scale and resolution is studied using
reconstructed dilepton invariant mass distributions of $\Zg$\ candidate events, and
the simulation is adjusted to match the data.  Uncertainties in the scale and resolution are used to
evaluate the systematic uncertainty due to these corrections.
The acceptance uncertainty from the lepton modeling is dominated
mostly by the electron selection efficiency uncertainty.

The jet energy scale (JES) and its uncertainty are derived by
combining information from test-beam data, LHC collision data and
simulation~\cite{jetcor}.
For jets within the acceptance, the JES uncertainty varies
%in the range $2-8\%$
in the range 4--8\%
as a function of jet $\pT$ and $\eta$. This uncertainty is higher than in the previous result~\cite{ATL-CONF-2011-034} because of the additional uncertainty due to multiple $pp$ interactions at high instantaneous luminosity. The jet energy
resolution and jet reconstruction/identification efficiency measured in data and in
simulation are in good agreement. The statistical uncertainties on the
comparisons,  10\%\ and 1--2\%\ for the energy resolution and the efficiency,
respectively, are taken as systematic uncertainties associated with these effects.
The effect on the acceptance is dominated by the JES uncertainty.

%\subsection{Systematic uncertainties on the simulated samples}
%\label{s:mcsyst}

The systematic uncertainty on the efficiency of the $b$-tagging algorithm has been estimated to be $6$\%\ for $b$-quark jets, based on $b$-tagging calibration studies using inclusive lepton and multijet final states.
The uncertainties on the tagging efficiencies for light and charm quarks are larger, but are not a significant source of uncertainty due to the
intrinsically high signal-to-background ratios in the dilepton final states.
The acceptance uncertainty due to $b$-tagging is about 3\% for all three channels.

The uncertainty in the kinematic distributions of the \ttbar\ signal events
gives rise to systematic uncertainties in
the signal acceptance, with contributions from the choice of generator, the modeling of initial- and final-state
radiation (ISR/FSR) and the PDFs.
The generator and parton-showering uncertainty (collectively labeled
`Generator' in Table~\ref{tab:syssum}) are evaluated by comparing the
{\sc MC@NLO} predictions  with those of
{\sc Powheg}~\cite{powheg,Frixione:2007vw,Alioli:2010xd} interfaced to
either {\sc Herwig} or {\sc Pythia}.
The uncertainty due to ISR/FSR is evaluated using the {\sc AcerMC} generator~\cite{Acer}
interfaced to the {\sc Pythia}\ shower model, and by varying the parameters controlling ISR
and FSR in a range consistent with those used in the Perugia Hard/Soft tune variations~\cite{Skands}.
Finally, the PDF uncertainty is evaluated using a range of
current PDF sets~\cite{cteq6l}.
The dominant uncertainties in this category of systematics are the
modeling of ISR/FSR and the generator choice.

The overall normalization uncertainties on the backgrounds from
single top quark and diboson production are taken to be
8.6\%~\cite{PhysRevD.83.091503} and 5\%~\cite{Campbell:2010ff},
respectively.
%
The systematic uncertainties from the background evaluations derived from the data include the statistical uncertainties in these methods as well as the systematic uncertainties arising from lepton and jet identification and reconstruction, and the MC estimates that are used.
An uncertainty on the data-driven $\Zg$+jets evaluation, based on the
expected $\met$ resolution in these events, is included by varying the
$\met$ cut in the control region by $\pm 5\GeV$ for $ee$ and $\mu\mu$
events, and $\pm 10\GeV$ in lepton+track events where the $\met$
resolution is poorer.
The systematic uncertainty on the fake identified lepton background prediction is 50\%, as measured using control regions with different flavor composition and photon conversion rates, as determined by Monte Carlo studies. A 20\% systematic uncertainty is set on the prediction of the fake track-lepton background, derived from a comparison of predicted and observed fake track leptons in control
regions defined as opposite-sign events with zero or one jet without an $\HT$ cut, and same-sign events with more than one jet.
%
The uncertainty on the measured integrated luminosity of the dataset
is 3.7\%~\cite{lumi}.

Table~\ref{tab:syssum} lists the contributions to the cross-section
measurement of each of the systematic uncertainties considered, in
percent, with the non-$b$-tag and the exclusive $b$-tag analyses
combined in the $ee$ and $\mu\mu$ columns.  Only non-$b$-tag events
are used in the $e\mu$ channel.  The relatively large `Generator'
uncertainty for $ee$ events is partly a result of the limited size
of the MC data set.  In the $e$TL and $\mu$TL columns the `MC
statistics' uncertainty is larger than in the $ee$, $\mu\mu$, and
$e\mu$ case because the number of events available is reduced by the
removal of events with two identified leptons.  The combined
uncertainty comes from the profile likelihood technique used to
determine the cross section~\cite{ATL-CONF-2011-034}, and takes into
account all correlations.  The statistical uncertainty is determined
by fixing all systematic uncertainties at their best-fit values in
the likelihood function.

% Table for 7 ch. combination
%
\begin{table*}[htb]
\begin{footnotesize}
\begin{adjustwidth}{-6em}{-6em}
\centering
\begin{tabular}{|l|c|c|c|c|c|c|}
\hline
           Uncertainties       $\Delta\sigma/\sigma$[\%]        & $ee$              & $\mu\mu$          & $e\mu$            &  $e$TL  & $\mu$TL   &Combined         \\ \hline

% Uncertainties (\%)       & \multicolumn{6}{|c|}{$\Delta\sigma/\sigma$[\%]} \\ \hline

Data statistics           &    $\pm 8.1$  &    $\pm 6.1$  &    $\pm 3.9$  &   $\pm 14.1$  &   $\pm 14.2$  &    $\pm 2.9$\\
\hline
Luminosity                &    +4.4/-3.8 &    +4.4/-3.9  &    $\pm 4.2$  &    +5.1/-4.2  &    +5.4/-4.4  &    $\pm 4.3$ \\
\hline
MC statistics             &    $\pm 1.6$  &    $\pm 1.2$  &    $\pm 0.8$  &    $\pm 5.5$  &    $\pm 4.6$  &    +0.7/-0.6 \\
Lepton uncertainties      &    +6.2/-5.4   &    +2.9/-1.3  &    $\pm 3.1$  &    $\pm 4.1$  &    +1.8/-1.6  &    +2.6/-2.2 \\
Track-leptons             &     ---          &    ---          & ---          &    $\pm 4.4$  &    $\pm 1.9$  &    +0.3/-0.2 \\
Jet/$\met$ uncertainties  &    +5.7/-5.7  &    +6.4/-3.5  & +4.7/-3.2  &    +14.8/-6.4  &   $\pm 13.1$  &    +4.4/-3.4 \\
$b$-tagging uncertainties &    +1.2/-1.0  &    $\pm 0.7$   &     ---          &     ---          &     ---          &    +0.4/-0.0 \\
$\Zg$+~jets evaluation    &    $\pm 0.4$  &    +0.5/-0.0  &      --- &    $\pm 6.2$  &    +2.4/-2.7  &    +0.3/-0.2 \\
Fake lepton evaluation    &    $\pm 3.3$  &    1.5/-1.3  &    $\pm 3.0$  &   $\pm 13.7$  &   $\pm 15.1$  &    $\pm 1.7$  \\
Generator                 &   +12/-11     &    +4.5/-4.3   &    +4.8/-4.5   &   +14/-11    &   +14/-13     &    +5.1/-4.9  \\
\hline
All syst.(except lumi.)    &   +16.4/-14.4  &    +8.8/-6.4  &    +8.2/-6.8  &   +27.9/-20.7  &   +26.5/-23.7   &    +8.0/-6.5 \\
\hline \hline
Stat. + syst.              &   +18.9/-16.9  &    +11.6/-9.5  &    +10.1/-8.8  &   +31.8/-25.2  &   +30.7/-27.8  &    +9.6/-8.2  \\
\hline

\end{tabular}
\caption{Overview of the \ttbar{} cross-section uncertainties.}
\label{tab:syssum}
\end{adjustwidth}
\end{footnotesize}
\end{table*}


\subsection{Cross-section measurement}
\label{s:nobtagging}

The expected and measured numbers of events in the signal region,
after applying all selection cuts for each of the individual
dilepton channels, are shown in Table~\ref{t:signal}. A total of
1920 candidate events are observed for the analysis without
$b$-tagging, and a total of 1400 candidate events are found for the
$b$-tag analysis.  There are 1221 events in common between the two
selections, and 179 exclusive $b$-tagged events.

In Fig.~\ref{f:ll_njets} the number of selected jets and the
expectation for $\lumitot$ are shown for the non-$b$-tag analysis
with the five channels combined, and for the $b$-tag analysis with
the three channels combined. In the non-$b$-tag case, all
requirements except the jet multiplicity selection are applied, and
in the $b$-tag case all requirements except the $b$-tag requirement
are applied. The $\met$ distributions for the combination of the
five non-$b$-tag and three $b$-tagged channels are
shown in Fig.~\ref{f:ll_met_ht}. All requirements except $\met$ are
applied.
The dominant backgrounds are $\Zg +$jets and $W$+jets
production with a fake lepton and, for the $b$-tag analysis,
single-top events.

\begin{figure*}
  \centering
   \subfigure[]{
    \includegraphics[width=0.45\textwidth]{pretag_combined_nJets.eps}
    }
   \subfigure[]{
    \includegraphics[width=0.45\textwidth]{btag_preTagFinalJetnJetJetProbTaggedJet_all_ATLAS2.eps}
    }
  \caption{ (a) Jet multiplicity distribution for
$ee$+$\mu\mu$+$e\mu$+$e$TL+$\mu$TL events without a $b$-tagging requirement. (b)
 Multiplicity distribution of $b$-tagged jets in the $ee$+$\mu\mu$+$e\mu$ channels. Contributions from diboson and
single top-quark events are summarized as `Other EW'. The
events in (b) are not a simple subset of those in (a) because the
event selections for the $b$-tag and non-$b$-tag analyses differ.
Uncertainties shown are statistical and systematic combined. The
distributions are shown as stacked histograms.}
  \label{f:ll_njets}
\end{figure*}

\begin{figure*}
  \centering
     \subfigure[]{
    \includegraphics[width=0.41\textwidth]{pretag_combined_Met_log.eps}
    }
   \subfigure[]{
    \includegraphics[width=0.41\textwidth]{btag_jetProbTaggedInSRMEt_all_logscale_ATLAS2.eps}
    }
  \caption{
The $\met$ distribution in the signal region for (a) the five
non-$b$-tag channels combined and, (b) the three $b$-tagged channels
combined.  Contributions from diboson and single top-quark events
are summarized as `Other EW'. Uncertainties shown are statistical
and systematic combined. The last bin in each figure is an
overflow bin, including all events above 190 GeV. The distributions
are shown as stacked histograms.}
  \label{f:ll_met_ht}
\end{figure*}

The cross-section results are obtained with a profile likelihood
technique, as described in Ref.~\cite{ATL-CONF-2011-034}. The
branching fraction for $t\rightarrow Wb$ is taken to be 100\% and
the acceptance is calculated for a top mass of 172.5 GeV.

The top-quark pair production cross section measured by combining
the seven channels, the non-$b$-tagged $ee$, $\mu\mu$, $e\mu$, $e$TL
and $\mu$TL and the exclusive $b$-tagged $ee$ and $\mu\mu$, is
\[
\sigmattbar=\mathrm{}\xsecctot \xseccstat
\mathrm{(stat.)}\xseccsyst \mathrm{(syst.)} \xsecclumi
\mathrm{(lumi.)}~\rm pb.
\]
Figure~\ref{fig:XsecSummary} summarizes the cross sections
for the individual channels,  and the combination of the non-$b$-tag and
the exclusive $b$-tagged data sets.

\begin{figure}[!h]
\centering
\includegraphics[width=0.5\textwidth]{summary_dilep_atlforaproval.eps}
    \caption{Summary of the individual cross section measurements and the combination of non-$b$-tag and
    exclusive $b$-tagged results. The vertical dashed line and yellow band are the approximate NNLO theory calculation and its uncertainty.}
    \label{fig:XsecSummary}
\end{figure}

The measured cross section is in good agreement with a similar
measurement made with 2010 data by the CMS
collaboration~\cite{Chatrchyan:2011nb}, %with 2010 ATLAS measurements
with an ATLAS measurement in the dilepton channel with earlier
data~\cite{ATL-CONF-2011-034}, and with the SM
prediction of
$165\pmasym{11}{16}$~pb. Compared to the earlier ATLAS measurement
in the dilepton channel, the statistical uncertainty of the
measurement has been reduced by a factor of four with the addition
of more data, and a small reduction in the systematic uncertainty,
which now dominates, has been achieved.



\subsection{Results}
\label{s:summary}


The top-quark pair production cross section is measured using  events
selected by requiring two oppositely-charged lepton candidates, at least two
additional jets and missing transverse energy.
%A measurement is also made requiring one of the jets to be identified as a $b$-quark jet.

The top-quark pair production cross section measured without
$b$-tagging is
\[
\sigmattbar=\mathrm{}\xsectot \xsecstat \mathrm{(stat.)}\xsecsyst \mathrm{(syst.)} \xseclumi \mathrm{(lumi.)}~\rm pb.
\]
Using $b$-tagging, the cross section is
\[
\sigmattbar=\mathrm{}\xsecbtot \xsecbstat  \mathrm{(stat.)}\xsecbsyst \mathrm{(syst.)} \xsecblumi  \mathrm{(lumi.)}~\rm pb.
\]
These results have been cross checked with other techniques,
confirming their robustness.
Finally the combination of the non-$b$-tagged and the exclusive $b$-tagged
data sets yields
\[
\sigmattbar=\mathrm{}\xsecctot \xseccstat  \mathrm{(stat.)}\xseccsyst \mathrm{(syst.)} \xsecclumi  \mathrm{(lumi.)}~\rm pb.
\]
The cross-section results are summarized in Fig.~\ref{fig:XsecSummary}.
%Along with the cross-checks, the kinematic properties of the selected events
%have been found to be compatible with a heavy particle pair where each particle decays into a jet, lepton and neutrino~\cite{ATL-CONF-2011-034}.

\begin{figure}[!h]
\centering
%    \includegraphics[width=0.5\textwidth]{f/summary_dilepDetail_atlprelim.eps}
\includegraphics[width=0.5\textwidth]{f/summary_dilepDetail.eps}
    \caption{ Cross section summary.}
    \label{fig:XsecSummary}
\end{figure}

The measured cross sections are in good agreement with a similar
measurement made with 2010 data by the CMS
collaboration~\cite{Chatrchyan:2011nb}, with 2010 ATLAS measurements
made in the complementary lepton+jets
channels~\cite{ATLAS-CONF-2011-023,ATLAS-CONF-2011-035}, with an
ATLAS measurement in the dilepton channel with earlier
data~~\cite{ATL-CONF-2011-034}, and with the SM prediction of
$165\pmasym{11}{16}$~pb. The agreement between the measurements with
and without $b$-tagging requirements confirms that the candidate
events arise from top quark pair production.
%Together,
%they set the stage for top production and decay studies with an
%order of magnitude more data that will be collected at the LHC in
%the near future.
