%% NYU PhD thesis format. Created by Jos� Koiller 2007--2008.

%% Use the first of the following lines during production to
%% easily spot "overfull boxes" in the output. Use the second
%% line for the final version.
%\documentclass[12pt,draft,letterpaper]{report}
\documentclass[12pt,letterpaper]{report}

%% Replace the title, name, advisor name, graduation date and dedication below with
%% your own. Graduation months must be January, May or September.
\newcommand{\thesistitle}{Proof of the Riemann Hypothesis}
\newcommand{\thesisauthor}{Jane Doe}
\newcommand{\thesisadvisor}{Professor Fulana de Tal}
\newcommand{\graddate}{May 2032}
%% If you do not want a dedication, scroll down and comment out
%% the appropriate lines in this file.
\newcommand{\thesisdedication}{To my dog Weierstra\ss, with affection.}

%% The following makes chapters and sections, but not subsections,
%% appear in the TOC (table of contents). Increase to 2 or 3 to
%% make subsections or subsubsections appear, respectively. It seems
%% to be usual to use the "1" setting, however.
\setcounter{tocdepth}{1}

%% Sectional units up to subsubsections are numbered. To number
%% subsections, but not subsubsections, decrease this counter to 2.
\setcounter{secnumdepth}{3}

%% Page layout (customized to letter paper and NYU requirements):
\setlength{\oddsidemargin}{.6in}
\setlength{\textwidth}{5.8in}
\setlength{\topmargin}{.1in}
\setlength{\headheight}{0in}
\setlength{\headsep}{0in}
\setlength{\textheight}{8.3in}
\setlength{\footskip}{.5in}

%% Use the following commands, if desired, during production.
%% Comment them out for final version.
%\usepackage{layout} % defines the \layout command, see below
%\setlength{\hoffset}{-.75in} % creates a large right margin for notes and \showlabels

%% Controls spacing between lines (\doublespacing, \onehalfspacing, etc.):
\usepackage{setspace}

%% Use the line below for official NYU version, which requires
%% double line spacing. For all other uses, this is unnecessary,
%% so the line can be commented out.
\doublespacing % requires package setspace, invoked above

%% Each of the following lines defines the \com command, which produces
%% a comment (notes for yourself, for instance) in the output file.
%% Example:    \com{this will appear as a comment in the output}
%% Choose (uncomment) only one of the three forms:
%\newcommand{\com}[1]{[/// {#1} ///]}       % between [/// and ///].
\newcommand{\com}[1]{\marginpar{\tiny #1}} % as (tiny) margin notes
%\newcommand{\com}[1]{}                     % suppress all comments.

%% This inputs your auxiliary file with \usepackage's and \newcommand's:
%% It is assumed that that file is called "definitions.tex".
%%
%% Place here your \usepackage's. Some recommended packages are already included.
%%

% Graphics:
\usepackage[final]{graphicx}
%\usepackage{graphicx} % use this line instead of the above to suppress graphics in draft copies
%\usepackage{graphpap} % \defines the \graphpaper command

% Indent first line of each section:
\usepackage{indentfirst}

% Good AMS stuff:
\usepackage{amsthm} % facilities for theorem-like environments
\usepackage[tbtags]{amsmath} % a lot of good stuff!

% Fonts and symbols:
\usepackage{amsfonts}
\usepackage{amssymb}

% Formatting tools:
%\usepackage{relsize} % relative font size selection, provides commands \textsmalle, \textlarger
%\usepackage{xspace} % gentle spacing in macros, such as \newcommand{\acims}{\textsc{acim}s\xspace}

% Page formatting utility:
%\usepackage{geometry}

%%
%% Place here your \newcommand's and \renewcommand's. Some examples already included.
%%
\renewcommand{\le}{\leqslant}
\renewcommand{\ge}{\geqslant}
\renewcommand{\emptyset}{\ensuremath{\varnothing}}
\newcommand{\ds}{\displaystyle}
\newcommand{\R}{\ensuremath{\mathbb{R}}}
\newcommand{\Q}{\ensuremath{\mathbb{Q}}}
\newcommand{\Z}{\ensuremath{\mathbb{Z}}}
\newcommand{\N}{\ensuremath{\mathbb{N}}}
\newcommand{\T}{\ensuremath{\mathbb{T}}}
\newcommand{\eps}{\varepsilon}
\newcommand{\closure}[1]{\ensuremath{\overline{#1}}}
%\newcommand{\acim}{\textsc{acim}\xspace}
%\newcommand{\acims}{\textsc{acim}s\xspace}

%%
%% Place here your \newtheorem's:
%%

%% Some examples commented out below. Create your own or use these...
%%%%%%%%%\swapnumbers % this makes the numbers appear before the statement name.
%\theoremstyle{plain}
%\newtheorem{thm}{Theorem}[chapter]
%\newtheorem{prop}[thm]{Proposition}
%\newtheorem{lemma}[thm]{Lemma}
%\newtheorem{cor}[thm]{Corollary}

%\theoremstyle{definition}
%\newtheorem{define}{Definition}[chapter]

%\theoremstyle{remark}
%\newtheorem*{rmk*}{Remark}
%\newtheorem*{rmks*}{Remarks}

%% This defines the "proo" environment, which is the same as proof, but
%% with "Proof:" instead of "Proof.". I prefer the former.
%\newenvironment{proo}{\begin{proof}[Proof:]}{\end{proof}}


%% Cross-referencing utilities. Use one or the other--whichever you prefer--
%% but comment out both lines for final version.
%\usepackage{showlabels}
%\usepackage{showkeys}


\begin{document}
%% Produces a test "layout" page, for "debugging" purposes only.
%% Comment out for final version.
%\layout % requires package layout (see above, on this same file)

%%%%%% Title page %%%%%%%%%%%
%% Sets page numbering to "roman style" i, ii, iii, iv, etc:
\pagenumbering{roman}
%
%% No numbering in the title page:
\thispagestyle{empty}
%
\begin{center}
  {\large\textbf{\thesistitle}}
  \vspace{.7in}

  by
  \vspace{.7in}

  \thesisauthor
  \vfill

\begin{doublespace}
  A dissertation submitted in partial fulfillment\\
  of the requirements for the degree of\\
  Doctor of Philosophy\\
  Department of Mathematics\\
  New York University\\
  \graddate
\end{doublespace}
\end{center}
\vfill

\noindent\makebox[\textwidth]{\hfill\makebox[2.5in]{\hrulefill}}\\
\makebox[\textwidth]{\hfill\makebox[2.5in]{\hfill\thesisadvisor\hfill}}
\newpage
%%%%%%%%%%%%% Blank page %%%%%%%%%%%%%%%%%%
\thispagestyle{empty}
\vspace*{0in}
\newpage

%%%%%%%%%%%%%% Dedication %%%%%%%%%%%%%%%%%
%% Comment out the following lines if you do not want to dedicate
%% this to anyone...
\vspace*{\fill}
\begin{center}
  \thesisdedication\addcontentsline{toc}{section}{Dedication}
\end{center}
\vfill
\newpage
%%%%%%%%%%%%%% Acknowledgements %%%%%%%%%%%%
%% Comment out the following lines if you do not want to acknowledge
%% anyone's help...
\section*{Acknowledgements}\addcontentsline{toc}{section}{Acknowledgements}
%% Write your acknowledgements in this file. If you do not want to acknowledge anyone,
%% you can delete this file and comment out the corresponding part in the "thesis.tex"
%% file.
%

This thesis would not have been possible without the help of many people.
I would like to primarily acknowledge the excellent mentoring of Kyle Cranmer, who has served as my thesis adviser for my graduate school career.
In addition, I would like to thank Allen Mincer and Peter Nemethy for their guidance, advice, and help throughout my time at New York University.
I am indebted to Akira Shibata for teaching me particle physics, computer programming, and for being an excellent example of a productive and successful scientist.
Much of the analysis done toward my degree (and throughout the ATLAS experiment) depended on his contributions to software and measurement techniques.
I would also like to thank Attila Krasznahorkay for the incredible amount of help, both technical and conceptual, that he has provided me over the past few years.
Aside from being an excellent mentor, he has developed and maintained much of the software used to process and analyze the incredible amount of data
produced by the ATLAS detector that made this thesis possible.
I would also like acknowledge Jonathan Zrake for always being available for a discussion, either long or short.
And I would also like to thank Dan Foreman-Mackey for helping to open up the world of computer science to me.
Finally, I would like to thank my parents and the rest of my extended family for their continual support.

\newpage
%%%% Abstract %%%%%%%%%%%%%%%%%%
\section*{Abstract}\addcontentsline{toc}{section}{Abstract}
%% Write your abstract here.
%
The Riemann Hypothesis has been among the most important open
problems in mathematics for the past 150~years. In this thesis
I solve it, providing a positive result.

\newpage
%%%% Table of Contents %%%%%%%%%%%%
\tableofcontents

%%%%% List of Figures %%%%%%%%%%%%%
%% Comment out the following two lines if your thesis does not
%% contain any figures. The list of figures contains only
%% those figures included withing the "figure" environment.
\listoffigures\addcontentsline{toc}{section}{List of Figures}
\newpage

%%%%% List of Tables %%%%%%%%%%%%%
%% Comment out the following two lines if your thesis does not
%% contain any tables. The list of tables contains only
%% those tables included withing the "table" environment.
\listoftables\addcontentsline{toc}{section}{List of Tables}
\newpage

%%%%% Body of thesis starts %%%%%%%%%%%%
\pagenumbering{arabic} % switches page numbering to arabic: 1, 2, 3, etc.
%% Introduction. If your thesis has no introduction, or chapter 1 is
%% meant to be the introduction, then comment out the lines below.
\section*{Introduction}\addcontentsline{toc}{section}{Introduction}

%% Write your introduction here.
%

%Before it was discovered in 1995 by the Tevatron, the existance of the top quark was nearly assured.

% To add:
% - The Standard Model
% - What isn't in the standard model
% - Results that need to be explained


The top quark is the most massive fundamental particle in the Standard Model of particle physics.
Discovered in 1995 at the Tevatron by both the CDF and D0 experiments, it was the final piece in the quark model described by the Standard Model theory of fundamental particle physics.
Long before it was confirmed experimentally, the existance of the top quark was assured as it was required for the consistency of the Standard Model.
Its presence is necessary to prevent anomalies and high energy divergences, to explain the measured properties of the couplings of the $b$-quark to the $Z$ boson, and to explain a number of precision electroweak measurements.

%The top quark is the most massive fundamental particle in the Standard Model of particle physics, including the recently discovered Higgs boson.
At 173 GeV \cite{PARTICLE_DATA_GROUP}, it weights about as much as a gold nucleaus and is significantly more massive than the next heaviest quark.
The reason is it so massive (even more massive than the recently discovered Higgs boson) remains unknown.
Its status as the most massive known fundamental particle has lead to speculation as to whether it plays a special role in electroweak symmetry breaking, or perhaps that its large mass is intimately tied to the solution of the hierarcy problem.
%http://arxiv.org/pdf/1206.4484v1.pdf

The top quark also plays an important role in many scenerios proposing new physical models.
Many proposed massive particles decay preferentially into top quarks or are the result of top quark decays.
Moreover, the production and subsequent decay of the top quark leads to an experimental signature that can be extremely useful in searches for new physics.
And many new physical models can be indirectly probed using the precision measurement of the properties of the top quark.
% http://www.osti.gov/accomplishments/documents/fullText/ACC0202.pdf

Because of its mass, the top quark decays almost immediately.
This means that it is the only quark whose properties can be directly measured, as it decays before it can before it can participate in low energy QCD processes that make the direct measurement of lighter quarks difficult.
% http://cms.web.cern.ch/news/precision-measurements-using-top-quarks-cms

The LHC is often described to as a ``top factory,'' referring to the incredible rate of top-quark production due to its high energy and luminosity beam.
The abundence of top quark events produced by the LHC allow for high precision measurements of the properties of the top quark.
The precise measurement of it's properties not only serves as a crucial test of the capibalities of the Large Hadron Collider and the ATLAS detector, but serves as an excellent way to search for Beyond the Standard Model physics.
Moreover, these measurements are an excellent application of state-of-the-art statistical techniques which have been developed to enable deep and precise measurements by the experiments at the LHC.
%Precise measurements of the Top Quark's properties require experimental and statstical techniques.

This thesis will describe two types of measurements that were used to study the Standard Model and to search for physical models proposed to explain beyond the Standard Model physics.
The first measurement is a sophisticated and precise determination of the top quark pair production cross-section at a center of mass energy of $\sqrt{s} = 7$ TeV.
The second is a search for exotic physical particles and signatures that lead to final states containing or resembling two or more top quarks.

%The top-quark pair-production cross-section has been previously measured at the Tevatron with the CDF and D0 experiments.
%The CDF result, using single-lepton decay channels, obtained a cross-section measurement at 1.8 TeV of $\sigma_{t\bar{t}} = 6.5^{+1.7}_{-1.4}$ pb. %http://arxiv.org/abs/hep-ex/0101036
%Similarly, D0, using nine decay channels, measured a cross-section of $5.69 \pm 1.21$ (stat) $\pm$ 1.04 (sys) pb, assuming a top quark mass of 172.1 GeV.

%The precise measurement of it's properties not only serves as a crucial test of the capibalities of the Large Hadron Collider and the ATLAS experiment, but serves as an excellent way to search for Beyond the Standard Model physics.

%Precise measurements of the Top Quark's properties require experimental and statstical techniques.

%% If your thesis has different "Parts", use commands such as the following:
%\part{First Part\label{part:one}}%
\chapter{Statement of problem\label{chap:one}}

In this chapter, \ldots

\section{The Riemann Hypothesis\label{sec:hypothesis}}

Blah, blah, blah. There is nothing interesting in
figure~\ref{fig:afigure}.
\begin{figure}[htb]
  \begin{center}
    \emph{This statement is false.}
  \end{center}%
\caption[This alternate caption appears in the list of figures.]{This is the
caption that appears under the figure. It may be quite long---you wouldn't want
such a long caption to appear in the ``list of figures''.}
\label{fig:afigure}
\end{figure}

More blah, blah. There is nothing interesting about
table~\ref{tab:atable} either.
\begin{table}[htb]
\caption[Strange rules.]{For some reason unfamiliar to me,
typesetting rules require one to place captions above tables,
but below figures. Go figure.}\label{tab:atable}
  \begin{center}
    \framebox{You could put a table here. I won't.}
  \end{center}
\end{table}

\section{Another section\label{sec:two}}

Notice that the fist paragraph is indented. There's a package
to do that automatically. Blah, blah. Blah, blah, blah, blah.

%\input{chap2} % further chapters -- change file names to meaningful things...
%\input{chap3}
%\part{Second Part\label{part:two}}%
%\input{chap4}
%\input{chap5}
%\input{chap6}
%%%%% Appendices start %%%%%%%%%%%%%%%%
%% Comment out the following line if your thesis has no appendix
\appendix
\chapter{One more comment\label{chap:append}}

This is an appendix.

%% Note: If your thesis has more than one appendix, NYU requires a "list of
%% appendices" page before the body of the thesis. I don't provide the tools
%% to create that here, so you're on your own for that one... Sorry.
%\input{app2}
%%%% Input bibliography file %%%%%%%%%%%%%%%
%% Bibliography. I didn't use BibTeX, so you're on your own if you'd like to do so.
%
\begin{thebibliography}{99}\addcontentsline{toc}{chapter}{Bibliography}
\bibitem{JBC91}J.~B.~Conway, \emph{Functions of One Complex
    Variable~I}. Second edition. Springer-Verlag, Graduate
    Texts in Mathematics~\textbf{11}, 1991.
\end{thebibliography}


\end{document}
