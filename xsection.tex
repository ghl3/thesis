%
%
%

\section{Top Quark Pair-Production Cross-Section Combination}

A precise measurement of the Top Quark Pair-Production Cross-Section is crucial for understanding the performance of the ATLAS detector, for testing predictions of the standard model, to searching for or constraining many new physical models.
In particle physics, a cross-section is a way to describe the rate of a specific interaction or class of interactions that is independent of the 


\subsection{Top Quark Pair-Production}

The primary means of producing top quarks at the LHC is via gluon fusion and subsequent decay into a pair of top quarks.

\begin{figure}
  \begin{center}

    \subfigure[Production]{
      % TopQuarkPairProductionDiagram: http://kjende.web.cern.ch/kjende/netzwerk/images/Feynman/WplusWminusBBar.png
      \includegraphics[width=.4\linewidth]{figures/xsection/TopQuarkPairProductionDiagram.png}
    }
    \subfigure[Decay]{
      % TopQuarkBranchingRatios: http://ej.iop.org/images/0034-4885/75/5/056201/Full/rpp347183f06_online.jpg
      \includegraphics[width=.4\linewidth]{figures/xsection/TopQuarkBranchingRatios.jpg}
    }
  \end{center}
  \caption{ A typical feynman diagram of top quark pair production at the LHC.  Top quarks decay into W-bosons and b-quarks nearly 100\% of the time.  The subsequent decays of the W-Bosons determine the event topology of the \ttbar event.  Diagram showing the branching ratios of Top Quark pairs into leptons and quarks.}
  \label{img:TopQuarkPairProduction}
\end{figure}

W bosons decay either leptonically, in which they produce a lepton and a neutrino, or hadronically, in which they produce a pair of quarks.
W's decay leptonically XX\% of the time, which consists of decaying via $W- \rightarrow e \bar{\nu_{e}}$ 10.75\% of the time, via $W- \rightarrow \mu \bar{\nu_{\mu}}$ 10.57\% of the time, and via $W- \rightarrow \tau \bar{\nu_{\tau}}$ 11.25\% of the time, and decay into quarks the other 67.60\% of the time. % [pdg]
% W-Boson: http://pdg.lbl.gov/2012/listings/rpp2012-list-w-boson.pdf
However, tau leptons are themselves unstable particles that subsequently decay either leptonically (muon 17.41 or electron 17.83), or hadronically 64.76\% of the time, with about 50\% of the tau's today decays into 1 hadron and 15\% into 3 hadrons.
% Tau pdg: http://pdg.web.cern.ch/pdg/2012/listings/rpp2012-list-tau.pdf
In the discussion that follows, we will consider the $W \rightarrow \tau  \rightarrow e$ or $W \rightarrow \tau  \rightarrow \mu$ to be leptonic decays of the top, and we will ignore the intermediate state $\tau$ in terms of our classification.
Hence, a pair of top quarks, which we assume always decay into a pair of W bosons, will decay into two leptons 6.5\% of the time (known as the ``dilepton'' channel), into a single lepton and a pair of quarks 34.4\% of the time (known as the ``single lepton'' channel), and entirely into quarks 45.7\% of the time (known as the ``all hadronic'' channel).
% pdg:
% Citation: J. Beringer et al. (Particle Data Group), PR D86, 010001 (2012) (URL: http://pdg.lbl.gov)



%% \begin{figure}
%%   \begin{center}
%%     % TopQuarkBranchingRatios: http://ej.iop.org/images/0034-4885/75/5/056201/Full/rpp347183f06_online.jpg
%%     \includegraphics[width=100mm]{figures/xsection/TopQuarkBranchingRatios.jpg}
%%   \end{center}
%%   \caption{}
%%   \label{img:TopQuarkBranchingRatios}
%% \end{figure}





\subsection{Lepton+Jets}


\subsection{Dilepton}


\subsection{Combination}



