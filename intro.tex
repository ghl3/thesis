
%% Write your introduction here.
%

%Before it was discovered in 1995 by the Tevatron, the existance of the top quark was nearly assured.

% To add:
% - The Standard Model
% - What isn't in the standard model
% - Results that need to be explained


The top quark is the most massive fundamental particle in the Standard Model of particle physics.
Discovered in 1995 at the Tevatron by both the CDF and D0 experiments, it was the final piece in the quark model described by the Standard Model theory of fundamental particle physics.
The top quark is the most massive fundamental particle in the Standard Model of particle physics, including the recently discovered Higgs boson.
At 173 GeV \cite{PARTICLE_DATA_GROUP}, it weights about as much as a gold nucleaus and is significantly more massive than the next heaviest quark.

The LHC is often described to as a ``top factory,'' referring to the incredible rate of top-quark production due to its high energy and luminosity beam.
The abundence of top quark events produced by the LHC allow for high precision measurements of the properties of the top quark.
The precise measurement of it's properties not only serves as a crucial test of the capibalities of the Large Hadron Collider and the ATLAS detector, but serves as an excellent way to search for Beyond the Standard Model physics.  
Moreover, these measurements are an excellent application of state-of-the-art statistical techniques which have been developed to enable deep and precise measurements by the experiments at the LHC.
%Precise measurements of the Top Quark's properties require experimental and statstical techniques.

This thesis will describe two types of measurements that were used to study the Standard Model and to search for physical models proposed to explain beyond the Standard Model physics.
The first measurement is a sophisticated and precise determination of the top quark pair production cross-section at a center of mass energy of $\sqrt{s} = 7$ TeV.
The second is a search for exotic physical particles and signatures that lead to final states containing or resembling two or more top quarks.

%The top-quark pair-production cross-section has been previously measured at the Tevatron with the CDF and D0 experiments.
%The CDF result, using single-lepton decay channels, obtained a cross-section measurement at 1.8 TeV of $\sigma_{t\bar{t}} = 6.5^{+1.7}_{-1.4}$ pb. %http://arxiv.org/abs/hep-ex/0101036
%Similarly, D0, using nine decay channels, measured a cross-section of $5.69 \pm 1.21$ (stat) $\pm$ 1.04 (sys) pb, assuming a top quark mass of 172.1 GeV. 

%The precise measurement of it's properties not only serves as a crucial test of the capibalities of the Large Hadron Collider and the ATLAS experiment, but serves as an excellent way to search for Beyond the Standard Model physics.  

%Precise measurements of the Top Quark's properties require experimental and statstical techniques.
