%
%
%


\section{ATLAS}
ATLAS is a multi-purpose particle detector designed for particle discovery.
In order to fulfull this broad role, ATLAS must excel in several ways:

\begin{itemize}
  \item It must be sensitive to a wide spectrum of particle that can be produced by particle colissions, including Leptons (Electrons and Muons), Hadrons (Pions, Kaons, and the many other bound states of quarks that typically come in the form of Jets), and Photons.
  \item It must be able to identify precisely the kinematic properties of these particles, which includes measuring their energies and directions with a high resolution.
  \item It must maintain its resolution over a large range of energies, roughly from 1 GeV up to and including the TeV scale.
  \item It must make these measurements at an extremely high rate, millions of times a second, and over the entire lifetime of the experiment, which could end up being decades
  \item It must be able to record, save, and successfully distribute its collected data to experimentalists and analyzers around the world.
\end{itemize}

These requirements are entirely non-trivial, and fulfilling them required decades of design, construction, engieneering, and maintenence.  
The original design for ATLAS
In order to be successful, ATLAS 
to use a wide spectrum of detector designed to discover particles and to precisely measure their proproperties.


Typical descriptions of the ATLAS detector list its various components and describe their properties.
But perhaps it is more useful to instead approach the detector from the perspective of the different particles that it is designed to detecct, and to use that as a springboard to go into machine's enieneering details.

\subsection{Jets}

The Large Hadron Collider is thus named because it collides a specific hadronic bound state of quarks known as the proton.
The majority of interactions between these particles are mediated by the strong force (QCD), and the vast majority of final state particles are cone-like sprays of hadronic particles known as Jets.
It is therefore crucial that a detector working with a hadron collider be excellent at identifying and measuring hadronic particles.

The primary means of studying a jet is through calrimetry.
The energy of a jet is measured by stopping the jet a component of the detector and measuring how much energy the jet deposits into that component.
The ATLAS hadronic calrimeter is designed 
