%
%
%


\section{ATLAS}
ATLAS is a multi-purpose particle detector designed for particle discovery.
In order to fulfull this broad role, ATLAS must excel in several ways:

\begin{itemize}
  \item It must be sensitive to a wide spectrum of particle that can be produced by particle colissions, including Leptons (Electrons and Muons), Hadrons (Pions, Kaons, and the many other bound states of quarks that typically come in the form of Jets), and Photons.
  \item It must be able to identify precisely the kinematic properties of these particles, which includes measuring their energies and directions with a high resolution.
  \item It must maintain its resolution over a large range of energies, roughly from 1 GeV up to and including the TeV scale.
  \item It must make these measurements at an extremely high rate, millions of times a second, and over the entire lifetime of the experiment, which could end up being decades
  \item It must be able to record, save, and successfully distribute its collected data to experimentalists and analyzers around the world.
\end{itemize}

These requirements are entirely non-trivial, and fulfilling them required decades of design, construction, engieneering, and maintenence.  
The original design for ATLAS
In order to be successful, ATLAS 
to use a wide spectrum of detector designed to discover particles and to precisely measure their proproperties.


Typical descriptions of the ATLAS detector list its various components and describe their properties.
But perhaps it is more useful to instead approach the detector from the perspective of the different particles that it is designed to detecct, and to use that as a springboard to go into machine's enieneering details.

\subsection{Layout and Geometry}

Cylander, eta, phi size, underground, lhc, geneva, Mount Blanc, etc

\subsection{Jets}

The Large Hadron Collider is thus named because it collides a specific hadronic bound state of quarks known as the proton.
The majority of interactions between these particles are mediated by the strong force (QCD), and the vast majority of final state particles are cone-like sprays of hadronic particles known as Jets.
It is therefore crucial that a detector working with a hadron collider be excellent at identifying and measuring hadronic particles.

The primary means of studying a jet is through calrimetry.
The energy of a jet is measured by stopping the jet a component of the detector and measuring how much energy the jet deposits into that component.
The ATLAS hadronic calrimeter is designed to contain as much of a hadronic shower within the calrimeter itself and to accurately determine its energy using the calrimeter's active components.
ATLAS' hadronic calrimeter system is divided into three subsystems that each cover different ranges in pseudorapidity.  The largest and most important is the barrel hadronic calrimeter, which is also simply known as the ``tile,'' which covers the rapidity region of $|\eta| < 1.7$. [tdr]  The hadronic end cap (HEC) extends the calrimeter's range up to $|\eta| < 3.2$.  Finally, the forward calrimeter (FCAL) covers the pseudorapidity range of $3.1 < |\eta| < 4.9$.  Each of these subsystems have similar goals but vary in their design.

The first step in measuring a jet's energy is stopping it.  The hadronic partilces that make up a jet are made up of quarks and gluons, which are charged under the strong force (QCD).
Therefore, they can interact with the nucleii of materials that they encounter.
The primary mechanism for energy loss of high energy hadrons traveling through a solid material is via inelastic nuclear interactions.
As these interactions cause the original hadron to lose energy, they will also cause nuclear excitations and subsequent nuclear decays in the material.
This in turn leads to the emission of additional hadronic particles, which themselves will interact with the material.
The net effect is the formation of what is known as a hadronic shower.
When the energy of the daughter particles in the shower becomes too small for nuclear excitation, the shower ceases to evolve, and the remaining energy of hadrons is lost to ionization and electromagnetic interactions.
While the majority of 

[pdg review: http://pdg.lbl.gov/2011/reviews/rpp2011-rev-passage-particles-matter.pdf]

[http://rd11.web.cern.ch/RD11/rkb/PH14pp/node80.html]


