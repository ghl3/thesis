%
%
%

\section{ATLAS}
ATLAS is a multi-purpose particle detector designed for particle discovery.
In order to fulfull this broad role, ATLAS must excel in several ways:

\begin{itemize}
  \item It must be sensitive to a wide spectrum of particle that can be produced by particle colissions, including Leptons (Electrons and Muons), Hadrons (Pions, Kaons, and the many other bound states of quarks that typically come in the form of Jets), and Photons.
  \item It must be able to identify precisely the kinematic properties of these particles, which includes measuring their energies and directions with a high resolution.
  \item It must maintain its resolution over a large range of energies, roughly from 1 GeV up to and including the TeV scale.
  \item It must make these measurements at an extremely high rate, millions of times a second, and over the entire lifetime of the experiment, which could end up being decades
  \item It must be able to record, save, and successfully distribute its collected data to experimentalists and analyzers around the world.
\end{itemize}

These requirements are entirely non-trivial, and fulfilling them required decades of design, construction, engieneering, and maintenence.  
The original design for ATLAS
In order to be successful, ATLAS 
to use a wide spectrum of detector designed to discover particles and to precisely measure their proproperties.


Typical descriptions of the ATLAS detector list its various components and describe their properties.
But perhaps it is more useful to instead approach the detector from the perspective of the different particles that it is designed to detecct, and to use that as a springboard to go into machine's enieneering details.

\subsection{Layout and Geometry}
Cylander, eta, phi size, underground, lhc, geneva, Mount Blanc, etc


\subsection{Overview}
ATLAS is a multi-component detector consisting of many individual subsystems, each of which measures a certain number of kinematic properties of particles created or scattered by the LHC.
Only when combining the measurements of these sub-systems do complete pictures of particles and entire begin to emerge.
The subcomponents of ATLAS are layered around the central location of the primary interaction.
The inner-most layer of ATLAS is the ``inner detector'', which consists of silicon strips and pixils and is designed to track the 3-d trajectories of charged particles passing through it.
The inner detector extends for 1.15 m and is surronded by a solonodial magent which provides a 2T field within the inner detector.
This field causes charged particles to bend and allows the inner detector to measure the momenta of charged particles by fitting their trajectories.
Surronding the inner detector is the electromagnetic calrimeter, which consists of alternating layers of lead absorbers and active electronics that are baithed in liquid argon.
The liquid argon must of course remain cooled, and so the electromagnetic calrimeter is located within a cryostat that maintains a temperature of [XX].
To reduce the amount of upstream material present before the calrimeter systems, the solonodial magent for the inner detector is located within the EM calrimeter's cryostat.
A dedicated presampler is located directly behind the cryostat's wall to correct for any energy lost within inactive material in front of the EM calrimeter.
Behind the outside of the cyrostat and past the EM calrimeter is the hadronic calrimeter, which consists of alternating layers of iron and scintillating tile.
Finally, the outer-most layer of ATLAS is the muon spectrometer, which consists of many subsystems designed to detect the trajectories of muons at high rates and with good resolution. 
A 8T magnetic field is induced across the entire muon system and is produced by large superconducting toroid magnets.


\subsection{Jets}
The Large Hadron Collider is thus named because it collides a specific hadronic bound state of quarks known as the proton.
The majority of interactions between these particles are mediated by the strong force (QCD), and the vast majority of final state particles are cone-like sprays of hadronic particles known as Jets.
It is therefore crucial that a detector working with a hadron collider be excellent at identifying and measuring hadronic particles.

% General Hadronic Calo layout
The primary means of studying a jet is through calrimetry, where the energy of a jet is determined by stopping a hadronic shower using a dense material and measuring the energy deposited using active material.
%The energy of a jet is measured by stopping the jet within a component of the detector and measuring how much energy the jet deposits into that component.
The direction of the jet is determined by the  $\eta$, $\phi$ location of where the jet hits the detector.
The ATLAS hadronic calrimeter is designed to contain as much of a hadronic shower within the calrimeter itself and to accurately determine its energy using the calrimeter's active components.
ATLAS' hadronic calrimeter system is divided into three subsystems that each cover different ranges in pseudorapidity.  
The largest and most important is the ``tile'' hadronic calrimeter (which itself is divided into the ``barrel'' and ``extended barrel'' tile calrimeters), which covers the rapidity region of $|\eta| < 1.7$. [tdr]  
The ``hadronic end cap'' (HEC) extends the calrimeter's range up to $|\eta| < 3.2$, and the ``forward calrimeter'' (FCAL) covers the pseudorapidity range of $3.1 < |\eta| < 4.9$.  
%Each of these subsystems have similar goals but vary in their design.

[pdg review: http://pdg.lbl.gov/2011/reviews/rpp2011-rev-passage-particles-matter.pdf]

[http://rd11.web.cern.ch/RD11/rkb/PH14pp/node80.html]

% The Tile
The hadronic tile calrimeter, which covers [XX] starradians, detects jets that appear close to perpendicular to the beam pipe.
It is a cylindric shell ranging from an inner radius 2.28 meters and outer radius 4.25 meters.
The tile is made of three subsystems, one central ``barrel'' calrimeter which covers $|\eta| < 1.0$ and two ``extended barrel'' calrimeters, which cover $0.8 < |\eta| < 1.7$ on either side.
The tile calrimeters consist of alternating layers of 14 mm iron plates, which is used for inducing the hadronic shower and stopping the jet, and 3mm active scintillating-tile, which is used to measure the energy deposited into the calrimeter by the jet.
%Iron is used because the hadronic particles that make up jets interact strongly with nucleii, and therefore a material with a high nuclear density is desirable.
The scintillating tile is attached to a series of photomultiplier tubes (PMTs) that amplify the tile's electrical readout.
Key to the successful performance of the tile calrimeter is providing enough average interaction lengths for the hadronic shower to be contained.
The hadronic thickness at the end of the tile calrimeter (at $\eta=0$) is 9.2 $\lambda$, which is enough to ensure good jet resolution and to minimize jet ``punch through'' into the muon system.



\subsection{Measuring Particles with ATLAS}

\subsection{Jets}
Jets are produced when particles with color charge, quarks and gluons, are in the outgoing states of a hard interaction.
As these particles propogate through space, they will tend to decay into or radiate other color-charged particles, leading to a phenomenon known as the evolution of a ``parton shower.''
This shower grows rapidly because QCD, which is an asymptotically free field theory, has a small coupling constant at high energies.  
As the shower progresses and the average energy of the constituent particles decreases, the QCD coupling constant will grow stronger, and this will cause the quarks and gluons to bind together into stable particles in a process known as ``hadronization.''
The collection of these hadrons, which move in a wide, cone-like shape, is known as a jet.
%The average distance for the parton-shower, hadronization evolution process is small compared to the radius of ATLAS, adn so jets are fully formed as they begin to interact with the detector

A jet originating from the beam spot and and moving through ATLAS will first interact with the inner detector.  Since many of the constituent particles of a jet are charged hadrons ($\pi^{+}$, $\K^{+}$, etc), they will leave tracks in the ID.
Jet algorithms can use these tracks when attempting to identify a jet, and there are algorithms that exclusively rely on this collection of tracks to build a set of identified jets (
[ Track Jets, 2009: https://atlas.web.cern.ch/Atlas/GROUPS/PHYSICS/CONFNOTES/ATLAS-CONF-2010-002 ].
However, the most common way to reconstruct jets is using the calrimetry.
Jets will deposit a small fraction of their energy in the Electromagnetic Calrimeter, but the majority of the shower's energy is measued by the hadronic calrimeter.
There are many algorithms used to reconstruct jets using information from the hadronic calrimeter, but most of them consist of clustering algorithms which group together cells of energy deposits.
The main differences between techniques involve how these energy deposits are calibrated (if at all) and how the cells are clustered (based on fixed geometries, such as cones, or based on iterative algorithms using a distance function to merge cell deposits).
A small number of jets managed to escape the hadronic calrimeter and enter the muon chambers, where they can be misidentified as mouns.
In addition, jets with ``heavy flavor'' may contain decays that result in real muons (but not muons that originated from a hard collission in the beam spot), so care must be taken to separate muons originating as jets from muons originating from the hard process.

% How jets work, and how they shower

%% The first step in measuring a jet's energy is stopping it.  
%% The hadronic partilces that make up a jet are made up of quarks and gluons, which are charged under the strong force (QCD).
%% They therefore interact with the nucleii of materials that they pass through.
%% The primary mechanism for energy loss of high energy hadrons traveling through a solid material is via inelastic nuclear interactions.
%% As these interactions cause the original hadron to lose energy, they will also cause nuclear excitations and subsequent nuclear decays in the material.
%% This in turn leads to the emission of additional hadronic particles, which themselves will interact with the material.
%% The net effect is the formation of what is known as a hadronic shower.
%% When the energy of the daughter particles in the shower becomes too small for nuclear excitation, the shower ceases to evolve, and the remaining energy of hadrons is lost to ionization and electromagnetic interactions.
%% While the majority of energy is lost via nuclear interactions, a non-neglegable fraction of a jet's energy interacts electromagnetically .
%% This fraction 


\subsection{Electrons}
Electrons and positrons are electromagnetically charge and wil therefore interact with the innner detector, and in addition their trajectories will be curved as the particles are beant by the magnetic field.
As they pass through the inner detector, electrons will deposit hits in the pixles, silicon, and straw tubes, and the collection of these hits can be extrapolated together to reconstruct the electron's track.
This track is of crucial importance for identifying electrons.
The track carries information both about the direction of the electron, but also about its momentum, and the presence of a track is necessary to distinuish electrons from photons, which lead essentially identical electromagnetic showers in the EM calrimeter.
When electrons enter the EM calrimeter, they will begin to decellerate, which will cause the electron to radiate via bremsstralung.
The radiated photons often have enough energy to produce electorn-positron pairs, which themselves will emit bremsstralung radiation.
This creates a cascade of electrons, positrons, and photons that is collectively known as an electromagnetic shower.
The energy of this shower is determined as it is absorbed by the EM calrimeter.
The EM calrimeter tends to provide a better energy resoltion for high pT electrons than the tracks do (beginning when the electron's energy is around 10 GeV).

[ 2010 performance: http://cdsweb.cern.ch/record/1273197 ]
[ https://atlas.web.cern.ch/Atlas/GROUPS/PHYSICS/PUBNOTES/ATL-PHYS-PUB-2011-006/ATL-PHYS-PUB-2011-006.pdf ]
[ Moriond 2012 Resolution info: https://indico.cern.ch/getFile.py/access?contribId=3&resId=1&materialId=slides&confId=163471]
[ https://twiki.cern.ch/twiki/pub/AtlasProtected/EnergyScaleResolutionRecommendations/summary_rel16.pdf ]

\subsection{Photons}
Photons are identified as electromagnetic showers that aren't matched to an inner detector track.
In addition, photons may decay into an electron-positron pair before entering the EM calorimeter.
If this occurs within or before the inner-detector, the proximity of the electron and positron's tracks can be used to identify the pair as a photon that ``converted'' electromagnetically, and the pair can be reconstructed as a single photon.
%In addition, photons may be identified as electron-positron pairs  from ``conversions,'' which are electron-positron pairs that 


\subsection{Muons}
Muons, with a mass of 106 MeV, weigh about 200 times more than the electron (which weights .512 MeV).
This implies that, for a fixed energy, a muon will have a smaller value of $\beta$ than an electron (or, equivalantly $\gamma$), which results in a smaller amount of radiation being emitted as it passes through matter.
Therefore, a muon's mean radiation length in ATLAS' calrimeters will be much longer than an electron's length, and a significant EM shower will not develop within the detector.
For this reason, most of a muon's energy escapes ATLAS.  

[ pdg: Muons through matter: http://pdg.lbl.gov/2000/passagerpp.pdf fig 23.1 ]

Since one can not measure its energy through calrimetry, ATLAS instead measures its momentum by estimating its curvature when bent in a magnetic field.
A Muon will create a bent track in the inner detector that can be used to estimate its momentum.
However, the resolution of the inner detector for high pt muons becomes somewhat poor.
To vastly improve on the measurement of the inner detector alone, ATLAS has a large set of detectors that comprise the muon spectrometer.
